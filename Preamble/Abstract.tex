
\section*{Abstract}
\addcontentsline{toc}{section}{Abstract}
\vspace{1cm}


% \vspace{1cm}

\begin{center}
	{\small This project aligns with the following CyBoK Skills: Network Security, Security Operations \& Incident Management }
	\vspace{10mm}
\end{center}

 \noindent In recent years, advancements in technology, such as machine learning, have seen a widespread increase in the reliance on computer systems for daily life. With this increased reliance, the complexity of cyber-attacks has increased. Conventional Intrusion Detection Systems (IDS) approaches have proven insufficient in detecting these emerging and advanced threats. Existing literature lacks the assessment of using machine learning in Wireless Network Intrusion Detection Systems (WIDS) to classify these using a combination of application layer features with 802.11 and non-802.11 network protocol features.

\smallskip

This project examines combining additional application layer features to train two ensembles (Random Forest \& XGBoost) and one neural network based (MLP) machine learning model for a proposed WIDS. The benchmark Aegean Wi-Fi Intrusion Dataset 3 (AWID3) was used, and six attacks (Botnet, Malware, SQL Injection, SSH, SSDP Amplification and Website Spoofing) were chosen to be classified. Models were evaluated on metrics such as AUC, F1 and Cross-Validation scores. The range of models, without relying on data balancing techniques, demonstrated high classification performances in all AUC, F1, Precision, Recall and Accuracy metrics of up to 99.9\%.

\vspace{2cm}

 \noindent \textit{Keywords: Application Layer Attacks, AWID3 dataset, MLP, Random Forest, Wireless Network Intrusion Detection Systems, XGBoost}