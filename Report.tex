% REPORT.TEX - University of Warwick Reports / Dissertations / Projects
% 
% Author - Chris Quinn 28/06/2020
% 
%
% A template for students and masters dissertations, flexible for your
% needs.
%
% This is the main .tex which will tell the compiler to include everything, 
% each chapter/section is then in folders for convenience, as you include more
% images it can get harder and harder to manage.
%

\documentclass[pdftex,11pt,a4paper,oneside]{article}
%Can change the pt, papersize etc.

\usepackage{amsmath} %For both in-line and equation mode
\numberwithin{equation}{section} %Numbering of our equations per section
\usepackage{algorithm}
\usepackage{algorithmic} %Algorithm styles, need to be nested for the example shown
\usepackage{fancyhdr} % For our headers
\usepackage{graphicx} % Inserting images
\usepackage{setspace} % Spacing on the front page for crest and titles
\usepackage[]{fncychap} % Styles can be Sonny, Lenny, Glenn, Conny, Rejne, Bjarne and Bjornstrup
\usepackage[hyphens]{url} %Deals with hyphens in urls to make them clickable
\usepackage{xcolor} %Great if you want coloured text
\usepackage{tabularx}
\usepackage{appendix}
\usepackage{caption}
\usepackage{booktabs} % Used for the classification report table
\usepackage{listings}
\usepackage{color}
\usepackage{subcaption}
\usepackage{graphicx}

% Bibliography
\usepackage[backend=biber,style=authoryear-ibid]{biblatex}
\setlength\bibitemsep{\baselineskip}

\definecolor{mygreen}{rgb}{0,0.6,0}
\definecolor{mygray}{rgb}{0.5,0.5,0.5}
\definecolor{mymauve}{rgb}{0.58,0,0.82}

\lstset{ 
  backgroundcolor=\color{white},   % choose the background color; you must add \usepackage{color} or \usepackage{xcolor}; should come as last argument
  basicstyle=\footnotesize, % the size of the fonts that are used for the code and the font family
  breakatwhitespace=false,         % sets if automatic breaks should only happen at whitespace
  breaklines=true,                 % sets automatic line breaking
  captionpos=b,                    % sets the caption position to bottom
  commentstyle=\color{mygreen},    % comment style
  deletekeywords={...},            % if you want to delete keywords from the given language
  escapeinside={\%*}{*)},          % if you want to add LaTeX within your code
  extendedchars=true,              % lets you use non-ASCII characters; for 8-bits encodings only, does not work with UTF-8
  frame=single,                        % adds a frame around the code on the top and bottom
  framexleftmargin=10pt,           % adjust the left margin of the frame
  framexrightmargin=10pt,          % adjust the right margin of the frame
  framextopmargin=5pt,             % adjust the top margin of the frame
  keepspaces=true,                 % keeps spaces in the text, useful for keeping indentation of code (possibly needs columns=flexible)
  keywordstyle=\color{blue},       % keyword style
  language=Octave,                 % the language of the code
  morekeywords={*,...},            % if you want to add more keywords to the set
  numbers=left,                    % where to put the line-numbers; possible values are (none, left, right)
  numbersep=5pt,                   % how far the line-numbers are from the code
  numberstyle=\tiny\color{mygray}, % the style that is used for the line-numbers
  rulecolor=\color{black},         % if not set, the frame-color may be changed on line-breaks within not-black text (e.g. comments (green here))
  showspaces=false,                % show spaces everywhere adding particular underscores; it overrides 'showstringspaces'
  showstringspaces=false,           % underline spaces within strings only
  showtabs=false,                  % show tabs within strings adding particular underscores
  stepnumber=1,                    % the step between two line-numbers. If it's 1, each line will be numbered
  stringstyle=\color{mymauve},     % string literal style
  tabsize=2,                       % sets default tabsize to 2 spaces
  title=\lstname                   % show the filename of files included with \lstinputlisting; also try caption instead of title
}

%KEEP THIS ONE LAST it's quite buggy, it allows you to click on links within the pdf and web links without changing the colour. The mouse cursor simply changes its icon to indicate to the user. Great tool - still awkward
\usepackage[hidelinks]{hyperref}


%This will tell the compiler to do the header style, page and spacing between the header and text
\fancyhf{}
\pagestyle{fancy}
\renewcommand{\headrulewidth}{0.2pt}


%%%%%%%%%%%%%%%%%%%%%%%%%% DOCUMENT STARTS %%%%%%%%%%%%%%%%%%%%%%%%%%%%%

\addbibresource{bibliography.bib}

%Lets begin the document, some chapters have examples in to give you an idea 
\begin{document}

% !TEX root =  ../Report.tex

\thispagestyle{empty}

\begin{spacing}{2}
	\begin{center}
		\includegraphics[scale = 0.45]{Preamble/WarwickCrest.pdf}
	\end{center}
	\vspace{5mm}
	\begin{center}
% 		\textbf{\begin{LARGE}
% 		Developing a Machine Learning Model To Detect Ransomware In Network Traffic
% 		\end{LARGE}}
        \textbf{\begin{LARGE}
        % Classification of IEEE 802.11w network attacks using Machine Learning Algorithms
        % Wireless Application Layer Attack-Based Intrusion Detection for IoT
        Detecting Application Layer Attacks on  IEEE 802.11 Networks Using Machine Learning
		\end{LARGE}}
		\vspace{5mm}
	\end{center}
	
	\begin{center}
		%{\large WM3B1 Cyber Security Project}\\
        \textbf{\large 2054584}
		\vspace{10mm}
	\end{center}
 
	\begin{center}
	     {\large Supervisor: Dr. Christo Panchev }\\
		\textbf{\large WMG Cyber Security Centre}\\
		{\large University of Warwick}\\
		{\large May 2023\\}
	\end{center} 
	
\end{spacing}

\pagenumbering{roman}



\section*{Abstract}
\addcontentsline{toc}{section}{Abstract}
\vspace{2cm}

% An abstract of no more than 250 words should clearly and concisely convey the aim, conduct and achievement of the project.

% \large
% When this thing gets up to 88 mph, you're gonna see some serious s$\cdot \cdot \cdot$.\\

% \vspace{1cm}

% \noindent \textit{Keywords: Flux Capacitor, 1.21 Gigawatts, Calvin Klein}

	\begin{center}
		{\small This project aligns with the following CyBoK Skills: Network Security, Security Operations \& Incident Management }
		\vspace{10mm}
	\end{center}
%\section*{Acknowledgements}
%\addcontentsline{toc}{section}{Acknowledgements}

\vspace{2cm}

\large

%I have to thank my supervisor Christo Panchev for being supportive and helping my along the way, to my family and friends who gave me the support through tough times, I just want to say: Thank you! %(yes, there are definitely people to acknowledge )
%Comment the whole thing out if you don't want it

\section*{Abbreviations}
\addcontentsline{toc}{section}{Abbreviations}
\large 

Aegan Wi-Fi Intrusion Dataset v3 \hfill AWID3 \\
Artificial Intelligence \hfill AI \\
Area Under Curve \hfill AUC \\
Autoencoders \hfill AE \\
Machine Learning \hfill ML \\
Intrusion Detection System \hfill IDS \\
Intrusion Prevention System \hfill IPS \\
Neural Network \hfill NN \\
Deep Neural Network \hfill DNN \\
Multi-Layer Perceptron \hfill MLP \\
K Nearest Neighbour \hfill KNN \\
Random Forest \hfill RF \\
eXtreme Gradient Boosting \hfill XGBoost \\
Address Resolution Protocol \hfill ARP \\
Domain Name Service \hfill DNS \\
Transmission Control Protocol \hfill TCP \\
User Datagram Protocol \hfill UDP \\
Server Message Block \hfill SMB \\
Secure Shell \hfill SSH \\
Simple Service Discovery Protocol \hfill SDDP \\
Stratified Cross Validation \hfill S-CV \\
F-Score/F-Measure \hfill F1 \\
Man-in-the-middle \hfill MITM \\
Denial Of Service \hfill DoS \\
\\

% Security Information And Event Management \hfill SIEM\\
% Security Orchestration, Automation and Response \hfill SOAR\\
% Endpoint detection and response \hfill EDR\\
% Extended detection and response \hfill XDR\\
\include{Preamble/Contents}
\include{Preamble/Lists}

\pagenumbering{arabic}

\lfoot{\centering Page \thepage}



% !TEX root =  ../Report.tex

\section{Introduction}
\label{sec:Introduction} 

The ongoing increase in IoT devices in homes and commercial environments has seen a
surge in wireless networks, particularly IEEE 802.11 networks, commonly referred to as Wi-Fi. As businesses and consumers seek to try out these new devices and technologies, manufacturers tend to prioritise improving performance and features, neglecting security \parencite{roundy_iot_nodate}. As a result, this may weaken the security posture of an organisation or home to be more susceptible to attacks from malicious threat actors taking advantage of vulnerable devices within a network.

\subsection{Wireless Networks And Attacks}
\smallskip

The 802.11 standards have advanced and improved since their inception in 1997 in terms of security; however, despite this, Wi-Fi networks are still vulnerable to well-known attacks such as de-authentication attacks (to disconnect all devices from a network), leading to more advanced attacks such as Man-in-the-middle attacks (MITM) or Evil Twin attacks \parencites{DBLP:journals/corr/abs-2105-15103}. The introduction of Protected Management Frames (PMF) in 2009 \parencite{5278657} helped to increase the security of management frames by using cryptography and integrity protection on de-authentication, disassociation and action management frames \parencite{9249426}. 

The introduction of WPA3 in 2018 \parencite{wifialliance_2022_wpa3} aimed to succeed WPA2, bringing new features and fixes to strengthen the security of wireless networks. More notably, Simultaneous Authentication of Equals (SAE) was introduced to provide a secure key negotiation and key exchange method based on the Dragonfly key exchange protocol in RFC 7664 \parencite{rfc7664}, preventing dictionary or brute-forcing attacks as well as the (KRACK) Key-Reinstallation attack \parencite{krack} by providing perfect forward secrecy, ensuring that even if the private key is obtained, the data packets cannot be decrypted. 

Research into WPA3 networks indicates that even features such as PMFs and SAE authentication methods have shortcomings, including being vulnerable to denial-of-service, side-channel, and downgrade attacks \parencite{vanhoef-sp2020-dragonblood}.


% \subsection{Background}
% \label{sec:Background}

\subsection{Intrusion Detection Systems}

\smallskip

Intrusion Detection Systems (IDS) are a common mechanism to defend against these attacks by analysing network traffic and determining if they are malicious or benign. There are typically two types of intrusion detection: signature-based and anomaly-based. Signature-based IDS monitors the network traffic for any suspicious patterns within data packets that match a known signature for an intrusion. This is usually via a database holding known intrusion attack patterns. Anomaly-based IDS creates an organisational benchmark of 'normal' as a baseline to help determine whether an activity is considered unusual or suspicious. This involves initially feeding the system with a large amount of data to learn an environment's regular usage patterns. 

External tools such as Stratosphere IPS (SLIPS) developed by \textcite{garcia_2015_slips} at the Stratosphere Lab at CTU University of Prague seek to utilise a combination of behaviour patterns and machine learning such as Markov Chain models to detect malicious network traffic. Open-source implementations of wireless IDS such as Kismet \parencite{kismet_2002_kismet} and OpenWIPS-ng \parencite{thomasdotreppe_2011_openwipsng} also exist and serve a usage for both consumers and businesses.

\medskip
Significant work and research have been seen recently investigating and developing wireless intrusion detection systems using machine learning-based algorithms utilising supervised, unsupervised and deep learning approaches in wired and wireless networks. However, research on Intrusion Detection Systems utilising 802.11 and other network protocol features, e.g. ARP, TCP \& UDP, including application layer features such as HTTP, DNS, SMB etc., lacks sufficient research.

This research seeks to investigate and evaluate different machine learning algorithms in detecting and classifying attacks launched at the application layer level on 802.11 wireless networks for a proposed wireless intrusion detection system. 

\subsection{Research Questions and Objectives}
\label{sec:Research Question}

The objectives for the project are as follows:
\begin{itemize}
\item To explore and analyse current literature and academic research utilising machine learning for intrusion detection systems for IEEE 802.11 networks.
\end{itemize}
\begin{itemize}
\item To examine and identify common machine learning algorithms used for the classification in the context of network attacks.
\end{itemize}
\begin{itemize}
\item To train a combination of supervised machine learning models to classify and detect a series of attacks launched from the application layer on 802.11 wireless networks.
\item To compare the performance of such models on the dataset, proving a recommendation for a proposed Wireless Intrusion Detection System (WIDS)
%\item How does combining 802.11 specific and non-802.11 with application layer features affect the detection of application layer attacks?
%\item To utilise a published data set to create machine learning models to help classify network attacks.
\end{itemize}

%\begin{itemize}
%\item To compare and evaluate the performance of the machine learning models to provide a recommendation based on accuracy, efficiency and suitability to help develop an Intrusion Detection System (IDS).
%\end{itemize}


\section{Literature Review}                               
\label{sec: Literature Review}

This section covers the existing research and reviews literature, papers and reports focusing on publicly available datasets, existing work and different machine learning algorithms. The literature reviewed details some of the methodologies and techniques used to develop existing models for detecting network attacks on 802.11 wireless networks. The following papers and literature inspire the practical element of this project.

\subsection{Datasets}

\textcite{9664737} discusses 37 public datasets, their suitability for building and training an IDS, and their limitations and restrictions. It was concluded that these datasets do not represent newer threats, such as zero-day attacks. An optimal dataset should consist of well-labelled, up-to-date, public network traffic ranging from regular user activity to attacks and payloads. It was proposed that using multiple datasets in different network environments and scenarios across a standard set of features could help to improve the accuracy of ML-based Network Intrusion Detection Systems.

\smallskip
The UNSW-NB15 dataset, \parencite{7348942} created by The University of New South Wales in Australia, is a well-known network intrusion detection dataset consisting of 49 features with nine attack classes, specifically: Analysis, Fuzzers, Worms, DoS, Reconnaissance, Generic, Exploits, Shellcode and Backdoors. It seeks to replace older datasets such as KDD98, KDDCUP99, and NSLKDD, frequently used to evaluate NIDS. However, the dataset was generated on non-wireless hardware and therefore did not align with the requirements of a wireless network dataset.

% !TEX root =  ../Report.tex

% About 3000 words probably

\smallskip
The recently published 5G-NIDD dataset \parencite{10.23729/e80ac9df-d9fb-47e7-8d0d-01384a415361} presents a labelled dataset built using 5G networks and contains a series of attack scenarios such as DDoS and port scans. As a relatively new dataset, it lacks existing literature and research for its utilisation for training an IDS. Moreover, being generated on 5G networks, it fails to meet this project's requirements of needing an 802.11w network dataset.
\medskip

The AWID3 dataset \parencite{9360747} released in February 2021 seeks to build upon the existing AWID2 dataset by evaluating various network attacks in an IEEE 802.11 enterprise network environment. These include higher-level layer attacks initiated from the link layer across multiple protocols and layers and newly discovered 802.11w attacks such as Krack, Kook, SSDP amplification, malware and even botnet attacks \parencite{kolias2015intrusion}. The dataset includes the Pairwise Master Key (PMK) and TLS Keys. Additionally, AWID3's concentration on enterprise networks includes Protected Management Frames (PMF) that help provide additional information during usage for an IDS. 

Previous work and research into evaluating numerous machine learning algorithms have been conducted on the well-known older AWID2 dataset \parencite{kolias2015intrusion}, however with an overall lack of publicly available wireless network datasets, the introduction of AWID3 can help to bring new research and training data to help develop new machine learning models.  

In the context of wireless networks, the AWID suite of datasets is widely recognised and used within academic research and literature; being one of the only extensive publicly available datasets on 802.11 enterprise networks concerning application layer attacks, AWID3 is a strong candidate for investigating the development of an IDS using machine learning. 

\subsection{Intrusion Detection Systems}

\textcite{10.4108/eai.27-11-2021.2315535} studies the performance of detecting 10 Denial Of Service attacks using Kismet on a Raspberry Pi using Aireplay-ng to generate a DoS attack on the target access point secured with WPA2/PSK, the experiment was repeated ten times. Using Kismet, the authors successfully identified the attack with an average detection time of 3.42 seconds.

\subsection{Detecting Network Attacks}

\subsubsection*{Application Layer Attacks}

\textcite{s22155633} discusses detecting application layer attacks using machine learning utilising the AWID3 dataset. The authors did not rely on optimisation or dimensionality-reducing techniques; only the six PCAPS containing application layer attacks were used, and more specifically, no application layer features were used, e.g. HTTP and DNS. These were classified and grouped under three main classes: Normal, Flooding and Other. This was justified because these are usually encrypted and, therefore, not easily accessible. Moreover, it raises privacy concerns, requiring attention to ensure the data does not contain personally identifiable information or data unique to the environment. A research gap was identified as no previous work focused primarily on detecting the attacks originating from the application layer on the newer AWID3 dataset.

A set of 802.11 and non-802.11 features was evaluated using three classifiers (Decision Tree, Bagging and LightGBM) and two DNNs (Multi-Layer Perceptron (MLP) and Denoising stacked Autoencoders (AE)). Bagging produced the highest-scoring AUC of the classifiers, with the MLP DNN performing slightly better than the AE across the non-802.11 and 802.11 features. The feature importance was evaluated, and irrelevant features were removed and tested in combination, resulting in better results across models. When the two feature sets were combined, the AUC saw a score of up to 95.30\%. Additionally, an 'insider feature' was engineered to represent if packets in the Botnet class are sent via client-client or client-server. This feature saw an improvement of up to 3\% in LightGMB and Bagging models. It is clear that this paper does not address the problem of using a set of application features or any optimisation techniques.  
\subsubsection*{5G Attacks}

\textcite{Mughaid2022} discusses the rise and need for protection from 5G-based attacks, including rule-based methods and machine learning-based methods. However, these methods have limitations in terms of accuracy and efficiency. To address these issues, the paper proposes a new system that leverages machine learning and deep learning techniques to achieve a high detection accuracy. 99\% accuracy was achieved using KNN and 93\% with DF and Neural Network.

\subsubsection*{Attack Classifications}

\textcite{10.1007/978-3-030-98457-1_1} utilised the AWID dataset to predict tuples of four attacks using the KNN classifier; the paper presented strong results for the "ARP" attack type, achieving the best accuracy with recall. The paper highlighted the importance of the pre-processing of data, feature selection, and choosing an appropriate classifier and oversampling method. The authors suggested that including additional features in the classification process and testing a more generalised model could improve a model's performance in future research and prevent the curse of dimensionality.

\medskip
The work by \textcite{DBLP:journals/corr/abs-2110-04259} investigates WPA3 Enterprise Networks against a combination of known WPA3 attacks alongside a series of older WPA2 attacks such as Beacon Flood and De-authentication attacks. It was concluded that eight of the nine attacks on the testbed's Access Point were vulnerable, and a chosen Intrusion Detection System could not identify and detect the attacks. \textcite{DBLP:journals/corr/abs-2110-04259} then designed a new signature-based IDS using Python. A packet capture of each attack was captured and processed into the proposed IDS. If there were indicators of attacks, the IDS outputted the time and classified the type of attack. The paper uses logical reasoning to deduce an attack rather than anomaly detection, such as machine learning.

\medskip

\textcite{pub.1154431160} investigated the detection of WPA3 attacks in the context of intrusion detection using their curated dataset based on existing and known WPA3 attacks: De-Authentication, RogueAP, Evil Twin, Krack and Beacon Flooding. A two-staged intrusion detection system is proposed. Traffic is first run through a Flood Detection system at the AP to detect sudden surges of data packets and secondly using an ML-based classifier. The data was trained on Logistic Regression, Decision Tree and Random forest and achieved a high accuracy of 99.98\% on Decision Tree and 99.97\% on Random Forest. 

\textcite{electronics12112355} utilised the AWID3 dataset for a proposed IDS for anomaly detection; the data was selected based on frames, and an equal number of frames were selected per class. Of the models tested, Decision Tree and Naive Bayes performed the best. Decision Tree achieved the best results with 98.57\% accuracy on the validation set and 96.79\% and 97.03\% accuracy on the custom testbed created with Beacon Flood and de-authentication attacks. The paper addresses the common issue of testing the created models on a data environment different from training.

In \textcite{s22041407} proposed an IDS capable of detecting DoS attacks on wireless sensor networks. Using the WSN-DS dataset, the K-Nearest Neighbor classifier was implemented with an arithmetic optimisation algorithm (AOA), and additionally, the Levy flight strategy was used for optimisation adjustment. The experiments concluded that the model reached up to 99\% accuracy, nearly a 10\% improvement from the original KNN algorithm.

The works by \textcite{10150/666297} utilised KNN on the AWID2 \& 3 datasets on ten features. To save memory, only the last thousand samples were used. The model quickly converged at a high accuracy of 0.95 on AWID2 and 0.88 on AWID3. 

\subsection{Machine Learning Algorithms}

The following table summarises a selection of existing literature and papers from the past five years related to the use of machine learning in detecting network attacks. The following common algorithms are abbreviated as follows: Random Forest (RF), Decision Tree (DT), Multi-Layer Perceptron (MLP), AutoEncoder (AE), Logistic Regression (Logreg), Neural Network (NN), Support Vector Machine (SVM), Naïve Bayes (NB) and K-Nearest Neighbour (KNN). It concludes that a wide array of machine learning algorithms have been utilised to detect network attacks. However, gaps still remain in using Random Forest and XGBoost on the AWID3 dataset.

\begin{table}[H]
\caption{Existing Literature Using ML Techniques}
\label{table:ml_papers}
\begin{tabular}{p{3cm}p{3cm}p{6cm}l}
\cline{1-3}
\textbf{Work}  & \textbf{Dataset/Data}  & \textbf{ML Methods} \\ \cline{1-3}
\cite{8455962} & MSU Scarda Dataset & \multicolumn{1}{p{6cm}}{\raggedright SVM, RF} \\ \hline
\cite{Ge2019DeepLI} & Bot-IoT & \multicolumn{1}{p{6cm}}{\raggedright Feed-Forward NN} \\ \hline
\cite{8746576} & AWID2 & \multicolumn{1}{p{6cm}}{\raggedright Ladder Network} \\ \hline
\cite{smys2020hybrid} & UNSW NB15 & \multicolumn{1}{p{6cm}}{\raggedright Hybrid Convolutional Neural Network} \\ \hline
\cite{9074929} & KDDCup99, NSL-KDD & \multicolumn{1}{p{6cm}}{\raggedright KNN, NB} \\ \hline
\cite{9249426} & AWID2 \& University of Arizona Dataset & \multicolumn{1}{p{6cm}}{\raggedright IsolationForest, C4.5, RF, AdaBoost, DecisionTable}   \\ \hline
\cite{DBLP:journals/corr/abs-2110-04259} & Mininet 2.2.2 & \multicolumn{1}{p{6cm}}{\raggedright SVM, MLP, DT, RF} \\ \hline 
\cite{s22155633}  & AWID3 & \multicolumn{1}{p{6cm}}{\raggedright DT, LightGBM, Bagging, MLP \& AE}   \\ \hline
\cite{pick_quality_over} & AWID 2 \& 3 & \multicolumn{1}{p{6cm}}{\raggedright Logreg, SGDClassifier, LinearSVC, LightGBM, DT, RF, Extra Trees, MLP, AE} \\ \hline
\cite{pub.1154431160} & AWID3 & \multicolumn{1}{p{6cm}}{\raggedright Logreg, DT, RF } \\ \hline
\cite{10.1007/978-3-030-98457-1_1} & AWID3 & \multicolumn{1}{p{6cm}}{\raggedright KNN} \\ \hline
\cite{Mughaid2022} & AWID3 & \multicolumn{1}{p{6cm}}{\raggedright DT, KNN, Decision Jungle, Decision Forest, Neural Network} \\ \hline
\cite{s22041407} & WSN-DS & \multicolumn{1}{p{6cm}}{\raggedright KNN} \\ \hline
\cite{DHANYA202357} & UNSW-NB15 & \multicolumn{1}{p{6cm}}{\raggedright RF, AdaBoost, XGBoost, KNN, MLP} \\ \hline
\cite{electronics12112355} & AWID3 & \multicolumn{1}{p{6cm}}{\raggedright DT, NB, RF, MLP } \\ \hline
\cite{10.1145/3508398.3519360} & AWID3 & \multicolumn{1}{p{6cm}}{\raggedright XGBoost, LightGBM, CatBoost} \\ \hline
\cite{su14148707} & X-IIoTDS, TON\_IoT & \multicolumn{1}{p{6cm}}{\raggedright XGBoost} \\ \hline
\cite{electronics9101689} & AWID2 & \multicolumn{1}{p{6cm}}{\raggedright Bagging, RF, ET, XGBoost, NB} \\ \hline
\end{tabular}
\end{table}



\medskip


\subsection{Summary}

Based on the literature review and research on the AWID3 dataset and wireless network attack classification, detecting application layer wireless network attacks using machine learning is under-researched. In their previous work, \textcite{s22155633} showed that combining 802.11 and non-802.11 features achieved high accuracy and AUC without using application layer features such as DNS, SMB and HTTP etc. However, it remains to be investigated whether combining these application layer features can improve the accuracy of machine learning classifiers in identifying application layer attacks on 802.11 networks. Furthermore, the works fail to classify the method of attack individually, combining the six attacks under three classes: Normal, Flooding and Other. This project aims to address this research gap by exploring the feasibility of using application layer features to enhance the performance of machine learning classifiers for detecting application layer attacks on the AWID3 dataset.

%!TEX root =  ../Report.tex

\section{Methodology}                               
\label{sec: Methodology}


\subsection{Ethics \& Risks}

Ethical approval was not required for this project and can be found in \ref{appx:Ethical Approval}. The following risks and ethical concerns are addressed as follows:
\begin{itemize}
	\item Data Reliability and Quality: Public datasets may vary in data quality and can lead to possibly unreliable results and conclusions. The chosen dataset is well-established and has extensive existing literature and research.
	\item Privacy Concerns: Datasets may contain personally identifiable information; however, in the context of this project and AWID3, features that may contain personal information will not be used for this project. 
\end{itemize}
In summary, no significant risks were identified, and no mitigations are required for this project.

\subsection{Code Environment}

The code for developing the machine learning models was programmed using Python 3.8/9, Visual Studio Code, and Jupyter Notebooks for the IDE. All experiments were conducted on a hardware combination of an M2 Mac Mini with 8 Cores and 16GB RAM or an Intel(R) Xeon(R) CPU E5-2699 VM running Ubuntu 22.04.02 LTS with 64 GB RAM and an Nvidia Tesla M40. Accordingly, the two machines will be referred to as 'M2' and 'VM'. Due to the Apple Silicon limitations and errors encountered, TensorFlow GPU Acceleration was not utilised for Deep Learning on the M2 Mac Mini.

\medskip
To create a reproducible environment and manage dependencies, Conda virtual environments \parencite{anaconda} were used to isolate the experiments on the M2 Mac Mini. A TensorFlow GPU docker container running Nvidia CUDA was utilised on the VM. See Appx \ref{appx: Conda_Env} for the complete code for creating the environments.

\subsection{Libraries}

Several libraries were used to develop and implement the machine learning models, including: 
A selection of common machine learning libraries was utilised for this project, namely Numpy, Pandas, Scikit-Learn \parencite{scikit-learn}, Matplotlib, Seaborn, Joblib, Jupyter, Tensorflow \parencite{tensorflow2015-whitepaper} and XGBoost \parencite{XGBoost}. 

\subsection{Feature Selection}

Similar to the work carried out by \textcite{s22155633}, six attacks out of the 13 from AWID3 were concentrated, namely Botnet, Malware, SSH, SQL Injection, SSDP Amplification and Website Spoofing; these are attacks that originate from the application layer and forms a good scope of research for this project. 

The following details relevant background information about each attack class \parencite{kolias2015intrusion}.
\begin{itemize}
	\item SSH Bruteforce - a brute-force attack was conducted against the radius server unsuccessfully for 180 seconds on the login credentials. 
	\item Botnet - The attack deployed pieces of malware within a Samba shared directory and assumed victims executed them. Four STAS were then infected, turning into bots. Remote commands were then executed, such as grabbing a screenshot of the desktop and sent to the attacker.
	\item Malware - Two pieces of malware were placed within a Samba share and downloaded by six STAs, but never executed. 
	\item SQL Injection - The target is an external node (DVWA), and a malicious SQL query string was inputted into a web form of the target. The packet's HTTP POST and GET requests can reveal the SQL code query.
	\item SSDP Amplification - This attack consists of a DDoS attack using the Simple Service Discovery Protocol. It uses Universal Plug and Play (UPnP) to trick all STAs of the wireless network into sending a barrage of packets to each SSDP-enabled device. Every device then responds, eventually leading to a DoS. In the dataset, the attacker scanned the intranet for ~30 seconds before launching the attack for 210 seconds on the DVWA webpage.
	\item Website Spoofing - The attack deployed a fake Instagram webpage and used ARP and DNS poisoning to redirect victims to the fake page, where entered credentials were stolen and decrypted. 
\end{itemize}




\subsection{Feature Selection}

Similar to the work carried out by \textcite{s22155633}, six attacks out of the 21 from AWID3 were concentrated, namely Botnet, Malware, SSH, SQL Injection, SSDP Amplification and Website Spoofing, these are attacks that originate from the application layer and forms a good scope of research for this project. 

This work aims to combine the (16) 802.11 and (17) non-802.11 features from \cite{s22155633} with a set of chosen application layer features with the aim to detect and classify the different application layer attacks. As previously established, existing research determined a high degree of accuracy and performance when combing both the 802.11 and non-802.11 features together, but a lack of research into determining if including additional application layer features would provide grounds for a further context into developing a machine learning model and affect its overall performance.

\subsubsection{Application Layer Features}

The AWID3 dataset contains 254 features within each of its attack CSV files, including application layer features in a decrypted format; provided by the decryption keys. While this may not be readily available in most cases, within an organization's internal network in the context of an IDS, some application layer features will be accessible, such as any unencrypted DNS, HTTP, SMB, and NBNS traffic since the keys to protected 802.11 wireless networks would be available. However, to ensure data privacy and avoid bias from information specific to the AWID3 environment or containing identifiable information such as URLs and IP addresses, these features were not selected for this study. Therefore, the selected application layer features can be seen in Table~\ref{tab:application_features}. By combining these selected application layer features, this study aims to develop a machine learning classifier capable of accurately distinguishing between the different types of wireless network attacks.

\begin{table}[H]
\centering
\begin{tabular}{lcc}
\hline
\multicolumn{3}{c}{\textbf{Application Layer Features (19)}} \\ \hline
Feature Name & Preprocessing Method & Data Type \\ \hline
nbns & OHE & object \\
ldap & OHE & object \\
dns & OHE & object \\
http.content\_type & OHE & object \\
http.request.method & OHE & object \\
nbss.type & OHE & int64 \\
smb2.cmd & OHE & int64 \\
http.response.code & OHE & int64 \\
ssh.message\_code & OHE & int64 \\
nbss.length & Min-Max & int64 \\
dns.count.answers & Min-Max & int64 \\
dns.count.queries & Min-Max & int64 \\
dns.resp.len & Min-Max & int64 \\
dns.resp.ttl & Min-Max & int64 \\
ssh.packet\_length & Min-Max & int64 \\ \hline
\end{tabular}
\caption{The selected set of application layer features.}
\label{tab:application_features}
\end{table}

The section below covers in more detail each of the selected features and their justification.

NetBIOS name service can be used to identify the names of machines on a network. The \textit{nbns} feature combined with the \textit{nbss.type} and \textit{nbss.length} can provide context into the connections made between machines on a network without including AWID3 specific information. Different types of session packets can be indicative of certain activities such as file transfers, remote execution etc. The length of the packets can also help to identify any anomalous activity that may be useful for a machine learning classifier. 

\medskip
\textit{http.content\_type, request.method and response.code}: These features relate to the HTTP used for web browsing. They can provide insights into the type of content accessed by an attacker, the type of request method used, and the HTTP response code that was received. These HTTP features can be used to help identify potential attacks exploiting web-based vulnerabilities such as SQL Injections or Website Spoofing.

\medskip
Domain Name System (DNS) is responsible for translating human-readable domain names to IP addresses. \textit{dns.count.answers, count.queries, resp.len, and resp.ttl} chosen can provide additional information about DNS traffic, such as the number of queries and answers, the response length, and the time to live of each response. These can be used to help identify potential reconnaissance attacks and provide insights into the network traffic patterns to  identify potential DNS-based attacks such as DNS spoofing, cache poisoning, or tunnelling.

\medskip
SMB (Server Message Block) is a client-server communication protocol used for sharing resources such as files and printers, in 2017 several Remote Code Execution vulnerabilities were discovered relating to the SMB protocol, including the wider known MS17-010 Eternal Blue exploit. By examining SMB activity, the \textit{smb.cmd} we can determine different access types such as SMB access attempts, SMB file transfers, or SMB authentication requests, using this it may be possible to identify anomalous behaviour that could be indicative of an attack. 

\subsubsection{802.11 Features}

The works by \cite{pick_quality_over} 

\begin{table}[hp]
\centering
\begin{tabular}{lcc}
\hline
\multicolumn{3}{c}{\textbf{802.11 Features (16)}} \\ \hline
Feature Name & \multicolumn{1}{l}{Preprocessing Method} & \multicolumn{1}{l}{Data Type} \\ \hline
radiotap.present.tsft & OHE & int64 \\
wlan.fc.ds & OHE & int64 \\
wlan.fc.frag & OHE & int64 \\
wlan.fc.moredata & OHE & int64 \\
wlan.fc.protected & OHE & int64 \\
wlan.fc.pwrmgt & OHE & int64 \\
wlan.fc.type & OHE & int64 \\
wlan.fc.retry & OHE & int64 \\
wlan.fc.subtype & OHE & int64 \\
wlan\_radio.phy & OHE & int64 \\ 
frame.len & Min-Max & int64 \\
radiotap.dbm\_antsignal & Min-Max & int64 \\
radiotap.length & Min-Max & int64 \\
wlan.duration & Min-Max & int64 \\
wlan\_radio.duration & Min-Max & int64 \\
wlan\_radio.signal\_dbm & Min-Max & int64 \\ \hline
\end{tabular}
\caption{The selected set of 802.11 features.}
\label{tab:802.11_features}
\end{table}

\subsubsection{Non-802.11 Features}

Table~\ref{tab:non80211} shows the non-802.11 features used in the analysis. It consists of Transport layer (TCP \& UDP) protocols features responsible for data transfer and ARP features that operate on the Data-link layer to resolve Mac addresses. By analysing 

\begin{table}[H]
\centering
\begin{tabular}{lcc}
\hline
\multicolumn{3}{c}{\textbf{Non-802.11 Features (17)}} \\ \hline
Feature Name & Preprocessing & Data Type \\ \hline
arp & OHE & object \\
arp.hw.type & OHE & int64 \\
arp.proto.type & OHE & int64 \\
arp.hw.size & OHE & int64 \\
arp.proto.size & OHE & int64 \\
arp.opcode & OHE & int64 \\
tcp.analysis & OHE & int64 \\
tcp.analysis.retransmission & OHE & int64 \\
tcp.checksum.status & OHE & int64 \\
tcp.flags.syn & OHE & int64 \\
tcp.flags.ack & OHE & int64 \\
tcp.flags.fin & OHE & int64 \\
tcp.flags.push & OHE & int64 \\
tcp.flags.reset & OHE & int64 \\
tcp.option\_len & OHE & int64 \\
ip.ttl & Min-Max & int64 \\
udp.length & Min-Max & int64 \\ \hline
\end{tabular}
\caption{The selected set of Non-802.11 Features}
\label{tab:non80211}
\end{table}

\subsection{Dataset Manipulation}

The AWID3 Dataset is supplied in two formats, a set of CSV files representing each method of attack and its subsequent data and the raw PCAP network captures. For this instance, the CSV files were utilised, and the dataset was manipulated to suit the purpose of experimentation. Each attack contained a folder with the data split into multiple CSV files; these needed to be rejoined to form one file/dataset so that it could be utilised and processed accordingly. 

\medskip
The methodology proposed was as follows:
\begin{enumerate}
    \item Combine all individual CSV files for each attack method into one file using a bash script.
    \item Import the file as a data frame and extract the desired features into a separate data frame.
    \item Remove Nan and fix invalid values
    \item Replace missing values to 0
    \item Remove Nan target values.
    \item Export the data frame as a new CSV file.
    \item Combine all reduced datasets into one large data set.
\end{enumerate}

\medskip

\textbf{Combining Files}

\smallskip
A bash script, Appendix \ref{appx: CSV Combiner Script} was created to list all contents of a given folder, containing the \textit{.csv} file extension and sorted into numerical order, i.e. 01, 02, 03. However, each file contained the CSV header, so only the first CSV file's header was read and written into the new \textit{'combined.csv'} file. All other files were read and appended into the new file, ignoring the first line, the CSV header. 

This step resulted in 6 large CSV files with the following rows and file sizes. See Table \ref{tab:full_data}

\begin{table}[H]
\centering
\begin{tabular}{llllll}
\cline{1-3}
\textbf{Class}  & \textbf{Rows} & \textbf{File Size} &  &  &  \\ \cline{1-3}
SSH              & 2,440,571     & 3 GB               &  &  &  \\
Botnet           & 3,226,061     & 4.27GB             &  &  &  \\
Malware          & 2,312,761     & 3.41GB             &  &  &  \\
SQL Injection    & 2,598,357     & 3.8 GB             &  &  &  \\
SSDP             & 8,141,645     & 8.02 GB            &  &  &  \\
Website Spoofing & 2,668,568     & 2.85 GB            &  &  &  \\ \cline{1-3}
\end{tabular}
\caption{Data Before Cleaning and Processing}
\label{tab:full_data}
\end{table}

\medskip

\textbf{Feature Extraction}

\smallskip
With the combined datasets, the selected features were extracted from the 254 features as referenced in Table \ref{tab:application_features}, \ref{tab:802.11_features} and \ref{tab:non80211}. Due to the large file sizes, numerous errors and kernel crashes were encountered while importing the file into Pandas. 

Instead of importing all columns, the required features were specified using the \textit{'use\_cols'} parameter along with the \textit{'chunk size'} parameter to read the file in smaller chunks to save memory and eventually combined them, forming one data frame. This saw a reduction in import time and lower memory consumption.

\medskip
\textbf{Data Cleaning}

\smallskip
The data was cleaned to ensure it was fit for the next data pre-processing stage. Rows that contained only NAN values were dropped, as well as missing Label values. Each column's missing/nan values were replaced and represented with 0, following a similar approach to \textcite{s22155633}.

\smallskip
Upon analysis, frequent hyphened values were observed, e.g. \textit{-100-100-10, 123-456-1, -10-2, 81-63-63 etc.}. These were more notable in the 802.11w features such as \textit{'radiotap.dbm\_antsignal'} and \textit{'wlan\_radio.signal\_dbm'}, this was expected, as being wireless radio features, \textit{'radiotap.dbm\_antsignal'} represents the signal strength in decibel milliwatts (dBm) and is captured via multiple antennas each representing the captured signal strength. A similar approach to \parencite{s22155633} was followed, extracting and keeping the first value in the sequence, e.g. \textit{-100-100-10} became -100, \textit{123-456-1} became 123, \textit{-10-2} became -10 and \textit{81-63-63} became 81. A regex expression was written to iterate through each column to replace these values accordingly.

\smallskip
Following on, invalid values were observed, such as the presence of values containing months such as: \textit{Oct-26, Oct-18, Feb-10} etc. This was proposed to be a processing error during the creation of the CSV files from the PCAP files and represented a low majority of the dataset. It was concluded that rows containing invalid values would be dropped from the data. A similar RegEX expression was written to filter out these values from the following columns: \textit{'tcp.option\_len', 'dns.resp.ttl', 'ip.ttl', 'smb2.cmd'}. The complete code for this section can be found in Appendix \ref{appx: Feature Extraction}.

\medskip

\textbf{Individual Datasets}

\smallskip
After data cleaning and processing, the final six individual data files consisted of the following. See Table \ref{tab:reduced_data}

\begin{table}[H]
\centering
\begin{tabular}{llllll}
\cline{1-3}
\textbf{Class}  & \textbf{Rows} & \textbf{File Size} &  &  &  \\ \cline{1-3}
SSH              & 2,433,851    & 298 MB               &  &  &  \\
Botnet           & 3,216,505     & 393 MB             &  &  &  \\
Malware          & 2,304,632     & 283 MB             &  &  &  \\
SQL Injection    & 2,590,119     & 317 MB             &  &  &  \\
SSDP             & 8,137,106     & 1.04 GB            &  &  &  \\
Website Spoofing & 2,666,406     & 340 MB            &  &  &  \\ \cline{1-3}
\end{tabular}
\caption{Data After Cleaning and Processing}
\label{tab:reduced_data}
\end{table}

\medskip

\textbf{Combining Datasets}

\smallskip
Finally, utilising the same bash script (\ref{appx: CSV Combiner Script}), the six reduced CSV files were combined into one large single data frame and then subsequently exported to a CSV file. The resulting file was 2.67GB in size and contained approximately 21,348,614 rows. 
\subsection{Pre-Processing Methods}

\subsubsection{Encoding}

One of the main decisions when building a model for a classification problem is the choice of encoding such as label, ordinal and one-hot encoding.

One-hot encoding was chosen to encode the categorical data for our models, a binary vector is created for each category, and at once only one element is set to 1 (referred to as 'Hot' i.e True) and the rest set to 0 (referred to as 'Cold' i.e. False). This approach will avoid assigning arbitrary numerical values to each variable that the model may interpret as having a weighting depending on its value. 

Ensemble Classifiers such as Random Forest do not require the target variable i.e. Labels to be encoded and can be interpreted as a string e.g. Normal, SSH, Malware etc. However, for deep learning, K-Nearest Neighbor and XGBoost we also utilised One-Hot Encoding to encode the target variable. Refer to \ref{appx:OHE Encoding} for the code used to One-Hot Encode the categorical features. 


\subsubsection{Normalisation}

Scaling was performed on the dataset for normalisation to help normalise all numerical values and bring features to a similar scale. MinMax scaler was chosen to scale the data between 0 and 1. As a linear scaling method, it helps preserve the original distribution's shape, ensuring it does not affect the underlying relationship between the different features in the data. Refer to \ref{appx:Scaling} for the code used to perform the MinMax scaler on the numerical features in the dataset.

\subsection{Data Balancing}

At its core, the dataset is imbalanced, with a majority of 'Normal' data with varying ranges of available malicious data from each attack class. Consideration was taken to utilise data balancing methods such as SMOTE and Random under/oversampling to help distribute the data. However, in a normal environment one would expect an overwhelming majority of Normal network traffic, therefore to best represent a real-life scenario, the data was kept imbalanced, ensuring changes were not made to the underlying distribution of the dataset. Refer to Table \ref{tab:split_data} for the distribution of each class respectively before and after splitting into the train and test sets.


\subsection{Data Split}

A stratified train-test split was performed on the dataset by splitting the entire dataset into training and testing sets to ensure the distribution of the target variable i.e. Label is the same in both sets. When training a machine learning model, the testing set is used to evaluate the performance of the model to help prevent overfitting. Overfitting can occur when the model learns all of the features and relationships of the training data; almost memorising the data. Subsequently, it struggles to predict new, unseen data.

\begin{table}[H]
\begin{tabular}{llll}
\textbf{Class}   & \textbf{Train Data (70\%)} & \textbf{Test Data (30\%)} & \textbf{Whole Data (100\%)} \\ \hline
Normal           & 10,668,482                       &   4,572,206           & 12,192,550                  \\
SDDP             & 3,849,896                &     1,649,955            & 4,399,881                   \\
Website Spoofing & 283,576            &     121,533       & 324,087                     \\
Malware          &  92,112                     &     39,476               & 105,270                     \\
Botnet           & 39,806                    &     17,060          & 45,493                      \\
SSH              & 8,317              &       3,565             & 9,506                       \\
SQL Injection    & 1,840                &     789                  & 2,103                  \\ \hline
\end{tabular}
\caption{Data Model Split into Train and Test Sets}
\label{tab:split_data}
\end{table}

Analysing the split, we observe a large imbalance of data between each class of attack, in particular, SQL Injection makes up less than 0.01\% of the entire dataset, with SSDP taking the majority 21\% of the data. 


\subsection{Cross Validation}

Due to the imbalanced nature of the datasets, stratified k-fold cross-validation with a k value of 10 was used, similar to the works carried out by \textcite{s22155633}. The training set will be split into 10 folds, the model is then trained on all folds, except one called the validation set. The model is then tested on the validation set for its performance metrics and recorded. This is then repeated for all 10 folds, so each fold is used as a test set. The results are then averaged to better represent the model's performance across the data. Stratified split ensures each fold contains the same proportion of samples within each class to preserve the underlying structure of the data. 

Finally, after Cross Validation, we train the model using the full training set and evaluate it based on the test set to obtain a final measure of performance, before finally saving the model.

\subsection{Machine Learning Algorithms}

A key area of the work was deciding the machine learning algorithms to use; a combination of classifiers and neural networks were considered in their context of suitability, efficiency and performance. A review of existing literature and research in Section \ref{sec: Literature Review} shows that a wide range of machine learning algorithms has been used for the purposes of classifying network attacks. There exists a research gap in the unexplored machine learning algorithms. As such, this study aims to explore the effectiveness of a few algorithms. AWID3 is a labelled dataset; as such, only supervised algorithms were used for this work. The following algorithms were employed due to their effectiveness in existing literature, reproducibility and the identified research gap in current research regarding their application to this specific task. These algorithms have been established and shown success in other machine-learning tasks and have been adopted within the existing literature; their implementation is readily available from well-known ML libraries such as TensorFlow, Sci-Kit learn and XGBoost \parencite{scikit-learn, tensorflow2015-whitepaper, XGBoost}. The models were coded using prior module knowledge and relevant libraries' documentation.


\subsubsection{Random Forest}

Random Forest is an ensemble learning algorithm combining multiple decision trees during its training process; at each node, the best features are selected to split the tree with additional pruning to help prevent overfitting. The individual decision trees' predictions are combined to make a final prediction.

\subsubsection{K-Nearest Neighbor}

K-Nearest Neighbor is a non-parametric algorithm that finds the k-closest neighbours to a given input. It classifies it based on the majority class within the k neighbours from a chosen metric, for example, the Euclidean distance. It is considered a more computationally intensive algorithm, requiring observing the training data during evaluation to make predictions. 

\subsubsection{XGBoost}

XGBoost, short for eXtreme Gradient Boosting, is a type of gradient-boosted decision tree. It was developed by \textcite{XGBoost} and is considered an efficient and scalable algorithm capable of handling large datasets and models. It utilises a collection, referred to as an ensemble, of decision trees combined to create a model capable of learning from the errors of the previous tree in a sequence. 

\subsubsection{Multi-Layer Perceptron}

A Multi-Layer Perceptron (MLP) works using a feed-forward artificial neural network that consists of an input layer, one or more hidden layers, and an output layer. Each layer within contains a given number of neurons connected to additional layers through weighted connections. During training, the gradient of the loss function (difference between the predicted values with the actual values) is calculated and the weights and biases are updated with an optimiser to ensure the model is able to generalise and learn from the data.
\subsection{Evaluation Metrics}

A key area of the work was deciding the specific metrics use to evaluate the performance of the models. Metrics are vital to determine if models were under or over-fitting on our data and helps to provide context into steps and modifications needed to improve the performances of our models. 
As a multi-class classification problem, we concerned on primarily two main metrics of evaluation: 


\subsubsection*{AUC-ROC}

The Area Under the Receiver Operating Characteristic Curve (AUC-ROC) measures the ability for a model to correctly distinguish between positive and negative classes. AUC-ROC is also insensitive to class imbalances. Similarly in the works carried in \parencite{s22155633};\parencite{pick_quality_over} AUC was used as one of the primary evaluation metrics.

\medskip

This value is first calculated by plotting the Receiver Operating Characteristic (ROC) curve using the True Positive Rate (TPR) against the False Positive Rate (FPR) for each classification thresholds. The TPR is measure of the proportions of positive values that were correctly classified. Similarly, the FPR is the proportion of negative values that are incorrectly classified as positive. Using the ROC curve, the area under the curve (AUC) is calculated. This value ranges between 0 and 1, where 0.5 represents at best random guessing, and 1 corresponds a perfect classifier.

\medskip

As our problem is multi-class, the AUC will be calculated by computing the one-vs-all metric for each class separately i.e,  calculated for each class individually, treating all samples for that class as positive and all other as negative. Then these scores are averaged to calculate a final AUC score.

\subsubsection*{F1}

The F1 score is a weighted average of both precision and recall. Precision is the fraction of correctly predicted positive instances out of all total predicted positive instances. Recall is the fraction of correctly predicted positive instances out of the total actual positive instances.

The F1-score was chosen due to its representation in an imbalanced dataset; as it considers both precision and recall. Accuracy can be a misleading metric %(\citealp[]{FAWCETT2006861}; \citealp[]{grandini2020metrics}). A model can predict the majority class i.e 'Normal' in most cases and still receive high accuracy, but in reality it poorly represents the minority classes.

\subsubsection*{Equations for Precision, Recall \& F1} 

\begin{equation*} Precision = \frac{True\ Positive}{True\ Positive + False\ Positive} \end{equation*}

\begin{equation*} Recall = \frac{True\ Positive}{True\ Positive + False\ Negative} \end{equation*}

\begin{equation*}
F_1 = 2 \cdot \frac{\mathrm{Precision} \cdot \mathrm{Recall}}{\mathrm{Precision} + \mathrm{Recall}}
\end{equation*}

\subsubsection*{Micro, Macro and Weighted}

In regular binary classification, metrics such as F1, Precision, Recall and AUC can be calculated easily, however for our multi-class classification problem a slightly different approach must be taken. In particular, there are three main methods:

\begin{itemize}
    \item Micro averaging uses the metric across all classes by counting the total true positives, false positives, and false negatives. This is the equivalent of using the accuracy i.e, fails to take into account class imbalances.
    \item Macro averaging calculates the metric in each class independently and then averages this for all classes, giving equal weight for all classes. It is used typically when all classes are equally as important, irrespective of the class size or any imbalances.
    \item Weighted averaging also calculates the metric for each class independently, but the average of the individual class scores are weighted with the number of samples in each class. It is used when performance across all classes are considered important, and the class imbalance needs to be considered.
\end{itemize}

Therefore, the weighted averaging method was chosen, leading to robust scores that takes into account both the number of samples within the class and its performance. It was observed that most previous works fails to mention the averaging method used for its evaluation metrics.

\subsubsection*{Classification Report}

In addition to viewing the averaged metrics across all classes, the classification report provides a comprehensive summary of detailing the metrics for Precision, Recall, Accuracy and F1 across each class. 

\subsubsection*{Confusion Matrix}

The Confusion Matrix is a table that displays the performance of a model by showing the number of true positives, false positives, true negatives and false negatives for each class. In other words, how accurate the classifier is on each class and how it tends to wrong predict each class for another (confusion). By examining the confusion matrix, we can identify any specific classes that may require additional tuning or changes to the model to improve its performance. Works by \citeauthor{pmlr-v29-Koco13} introduced a new method using confusion matrices to measure and analyse the performance of cost-sensitive methods showing the importance of the confusion matrix in imbalanced data sets.


 %!TEX root =  ../Report.tex

\section{Experiments}
\label{sec: Experiments}

\subsection{Initial Modelling}
In order to speed up initial training and testing for each machine learning algorithm, a multitude of subsets of the original combined data were created using sklearn's train\_test\_split to create a stratified split resulting in reduced data sets. Varying levels of data splits were created, including a 50\%, 60\% and 80\% data split from the original ~12 million rows of data as seen in Table \ref{tab:split_data}. Furthermore, due to the limitations in hardware and training time, the majority of models were optimised on a trial and error methodology to find the optimal parameters, with some instances where GridSearchCV was used. We did not use 10 fold Stratified Cross Validation during initial experiments, this was justified under the pretense that the best found parameters would undergo cross validation training at a later stage. 

\subsection{Classifiers}

Classification on the combined set of features was conducted against three classifier models, Random Forest (RF), XGBoost and K-Nearest Neighbour (KNN).
%!TEX root =  ../Report.tex
% %Here are some subsections so that they will appear on the contents
\subsection{Classifiers}

%Classification on the combined set of features was conducted against three classifier models, Random Forest (RF), XGBoost and K-Nearest Neighbour (KNN).

\subsubsection{Random Forest (RF)}
\label{sec:RF}

Basically existing research doesn't have any models or research using a random forest ensemble classifier, only decision trees in \parencite{9360747}

\begin{table}[h]
\centering
\caption{Parameters for Random Forest Classifier}
\label{tab:rf-params}
\begin{tabular}{|l|l|l|l|}
\hline
\textbf{Parameter} & \textbf{Description} & \textbf{Value}  \\ \hline
n\_estimators & The number of trees in the forest & 100\\
criterion & Function to measure the quality of a split. & gini\\
max\_depth & Maximum depth of the tree. &  None \\
min\_samples\_split & Minimum samples required to split an internal node. & 2 \\ 
min\_samples\_leaf & Minimum samples required to be at a leaf node. & 1 \\
max\_features & Maximum features to consider when splitting.  & auto \\
bootstrap & To bootstrap samples when constructing trees & True \\
class\_weight & Weights associated with classes & None  \\
random\_state & The random seed & 1234 \\ \hline
\end{tabular}
\end{table}

% \begin{table}[h]
% \centering
% \caption{RF Model Metrics}
% \label{tab:rf-metrics}
% \begin{tabular}{|l|l|l|l|l|l|l|}
% \hline
% \textbf{Model} & \textbf{Data Subset} & \textbf{Accuracy} & \textbf{Precision} & \textbf{Recall} & \textbf{F1} & \textbf{Time} \\ \hline
% Base & 80\% & 0.997 & 0.955 & 0.882 & 0.916 & 00:02:34:59 \\ \hline
% Base & 100\% & 0.997 & 0.997 & 0.997 & 0.997 & 00:00:30:60 \\ \hline
% \end{tabular}
% \end{table}

\begin{table}[h]
\centering
\caption{RF Model Metrics}
\label{tab:rf-metrics}
\begin{tabular}{|l|l|l|l|l|l|}
\hline
\textbf{Model} & \textbf{Data Subset} & \textbf{Accuracy} & \textbf{Precision} & \textbf{Recall} & \textbf{F1}  \\ \hline
Base & 80\% & 0.997 & 0.955 & 0.882 & 0.916  \\ \hline
Base & 100\% & 0.997 & 0.997 & 0.997 & 0.997 \\ \hline
\end{tabular}
\end{table}

\begin{table}[h]
\centering
\caption{RF Model v2}
\label{tab:rf-model2}
\begin{tabular}{|l|l|l|l|l|l|l|l|}
\hline
\textbf{Device} & \textbf{Model} & \textbf{Data Size} & \textbf{AUC} & \textbf{Precision} & \textbf{Recall} & \textbf{Accuracy} & \textbf{F1}  \\ \hline
GPU & Optimised & 100\% & 99.99 & 99.66 & 99.67 & 99.67 & 99.66 \\ \hline
\end{tabular}
\end{table}

\paragraph{Confusion Matrix}

\begin{figure}[H]
    \centering
    \includegraphics[width=0.85\textwidth]{Appendices/NN Confusion Matrix 3-04-23.png}
    \caption{RF Confusion Matrix}
    \label{fig:rf_confusion_matrix}
\end{figure}

 \newpage

\subsubsection{XGBoost}
\label{sec:xgboost}

A series of models were trained and this model followed both an exploratory and iterative approach, RandomizedSearchCV was also utilised in the training phase to optimise hyperparameters. Table \ref{tab:xgboost-parameters} shows the parameters used for each model, \textbf{-} denotes that no value was used, i.e., the default value was used.

\begin{table}[h]
  \centering
  \caption{XGBoost Model Parameters}
  \label{tab:xgboost-parameters}
  \begin{tabular}{lcccccc}
    \toprule
    Parameter & Model 0-2\&6 & Model 3 & Model 5 & Model 8 & Model 10 & Model 11 \\
    \midrule
    early\_stopping\_rounds & - & 10 & 10 & 10 & 10 & 10 \\
    subsample & - & - & - & - & 0.9 & - \\
    n\_estimators & - & - & 300 & 300 & 200 & 300 \\
    min\_child\_weight & - & - & - & - & 3 & - \\
    max\_depth & - & - & 5 & 5 & 9 & 5 \\
    learning\_rate & - & - & 0.2 & 0.2 & 0.3 & 0.2 \\
    gamma & - & - & - & - & 0 & - \\
    colsample\_bytree & - & - & - & - & 0.7 & - \\
    reg\_alpha & - & 0.1 & 0.1 & 0.1 & - & 0.1 \\
    \bottomrule
  \end{tabular}
\end{table}


\medskip

Initial experimentation began with the XGBoost classifier being trained with default parameters across a range of subsets of data, 60\%, 80\% and 100\% as shown in models 0, 1 and 6. Additionally, model 1 further incorporated Stratified Cross Validation during training. This was used to establish clear baseline performance of the classifiers and helped to provide context when tuning parameters.

Where available, the VM machine was used for training, allowing XGBoost to maximise its performance on the dedicated GPU. Early stopping and regularisation were added in subsequent models to help avoid the model from overfitting on the training data. e.g. in Model 3 only early stopping and regularisation were added and no noticeable performance increase/decrease was observed. 

\subsubsection*{Parameter Tuning}

An attempt was made to utilise GridSearchCV for parameter optimisation; significant setbacks were encountered in achieving a successful execution due to numerous errors, system crashes, and exceptionally high grid search time. In light of the difficulties faced with GridSearchCV, RandomizedSearchCV was utilised. 

Despite challenges faced with GridSearchCV, initial experimentation with 5 CV on a smaller parameter grid on the 80\% dataset was successful and subsequent parameters were tested with models 5 and 8. An additional RandomizedSearchCV was run on the entire dataset and combined with stratified 10-fold cross-validation to ensure the best-found parameters were verified and proved consistent across the 10 folds.

See Listing \ref{lst:param_grid_xgb} for the parameter grids used.

\medskip
	
\begin{lstlisting}[language=Python, caption={Grid Search Parameters For XGBoost}, label= lst:param_grid_xgb]

gscv_param_grid = {
    'learning_rate': [0.05, 0.1, 0.2],
    'n_estimators': [100, 200, 300],
    'max_depth': [3, 4, 5]
}

rgs_param_grid = {
    'learning_rate': [0.01, 0.1, 0.3],
    'max_depth': [3, 6, 9],
    'min_child_weight': [1, 3, 5],
    'gamma': [0, 0.1, 0.2],
    'subsample': [0.5, 0.7, 0.9],
    'colsample_bytree': [0.5, 0.7, 0.9],
    'n_estimators': [100, 200]
}
\end{lstlisting}

\begin{table}[H]
\captionsetup{justification=centering} 
\centering
\caption{XGBoost GridSearch Parameters}
\begin{tabular}{lll}
\hline
\textbf{Parameter} & \textbf{GS} & \textbf{RGS} \\ \hline
Early Stopping & - & 10 \\
Evaluation Metric & merror & merror \\
Learning Rate & 0.2 & 0.3 \\
Max Depth & 5 & 9 \\
Min Child Weight & - & 3 \\
Gamma & - & 0 \\
Subsamples & - & 0.9 \\
Colsample By Tree & - & 0.7 \\
N Estimators & 300 &  200 \\ \hline
\end{tabular}
\label{tab:xg_gs_parameters}
\end{table}

\textbf{Best Found Parameters}
\medskip

The parameter tuning process helped to identify a set of optimal parameters using RandomisedSearchCV, as detailed in Table \ref{tab:xg_gs_parameters}. These parameters were used to create Model 10. Subsequent models created afterwards, from additional parameter tuning did not show a sign of noticeable improvement in the model's performance. Due to limited time and computational power, further parameter tuning was not performed, and the decision was made here to stop further experiments.

\subsection{Neural Networks}
\label{sec: Neural Networks}

\subsubsection{Multi-Layer Perceptron (MLP)}
\label{sec: MLP Neural Network v1}

As part of the neural network experiments, Multilayered Perceptron models were created and tested through an exploratory process. It should be noted a wide range of MLP models was explored, the selection presented in the results section is a curated list, chosen for performance or notability. All models evaluated and their corresponding code can be found within the codebase. Table \ref{tab:mlp-models-1} and \ref{tab:mlp-models-2} details the set of parameters used for each notable MLP model.  

Experimentation began with a 3 hidden layered MLP model consisting of 128, 64 and 32 neurons across the different subsets of data to gauge a rough estimate of the model's performance through varying levels of data. As such, cross-validation was not utilised. Models 0-3 consist of the same parameters tested across the 60 and 80\% datasets, metrics were high, however, the models struggled to predict minority classes and resulted in low recall and F1-Scores. Performance when increasing the size of the dataset did improve the performance and can be attributed to the fact the larger dataset provided more samples of the minority class to be trained on. With this information, further models were created with increased batch sizes and epochs to train for longer.

\subsubsection*{Overfitting}

A key aspect when training the MLP models was to prevent overfitting. To help mitigate this, techniques such as Early Stopping and Dropout were used in the majority of the models. Early Stopping was used during SCV, the training process monitors the validation AUC loss for signs of overfitting (e.g. when the model starts to learn the data and not generalising). Once the validation loss started to degrade over two defined epochs, the model would stop training. Dropout is a regularisation method that randomly sets 0.2 of the input neurons to 0. Dropout layers were used in the network architecture.

 
\subsubsection*{Thresholds}
Towards the latter stages of experimentation, an attempt was made to further enhance the performance of the models on the misclassified classes. The individual class weightings were adjusted using the thresholds of each class. The aim was to identify the optimal threshold level between 0-1 that would maximise the F1 score for that class. A systematic approach was followed to adjust the value in the class and evaluate the confusion matrices for changes in predictions.
 
 
\subsubsection*{Activator}
Due to the nature of the problem (multi-class classification), applying existing knowledge and experience, the softmax activator was chosen for the output layer. It provides an easy-to-interpret output of the model as a list of probabilities for each class and uses the highest probability as the predicted class.

\subsubsection*{Tuning}

The device used to train and the experiment was the M2 Mac Mini, experiments conducted on the VM were found to be slower and would frequently cause crashes, even when utilising the dedicated GPU. As such, the hardware and time constraints restricted the level of tuning and parameter searching that could be performed. Techniques such as GridSearchCV and RandomisedSearchCV were not feasible when combined with 10 Fold S-CV. 

\medskip
Due to the complexity and computational demands of running machine learning models, practical limitations such as time constraints result in fewer tested models than desired. After conducting a vast amount of experiments and achieving high-performance results, the decision was made to conclude further model experimentation. 
 
\medskip
\begin{table}[H]
\centering
\caption{MLP Model Parameters Part 1}
\label{tab:mlp-models-1}
\begin{tabular}{llll}
\hline
Parameter & Model 0-3 & Model 4 & Model 5 \\ \hline
Asctivator: & ReLU & ReLU & ReLU \\
Output Activator: & Softmax & Softmax & Softmax \\
Initialiser: & he\_uniform & he\_uniform & he\_uniform \\
Optimiser: & Adam & Adam & SGD \\
Momentum: & N/A & N/A & N/A \\
Early Stopping: & N/A & 2 & 2 \\
Dropout: & 0.2 & 0.2 & 0.2 \\
Learning Rate: & 0.001 & 0.001 & 0.01 \\
Loss: & CC & CC & CC \\
Batch Norm: & True & True & True \\
Hidden Layers: & 3 & 3 & 4 \\
Nodes per Layer: & 128/64/32 & 128/64/32 & 256/128/64/32 \\
Batch Size: & 180 & 200 & 132 \\
Epochs: & 15 & 20 & 20 \\ \hline
\end{tabular}
\end{table}

\begin{table}[H]
\centering
\caption{MLP Model Parameters Pt2}
\label{tab:mlp-models-2}
\begin{tabular}{lll}
\hline
Parameter         & Model 6       & Model 7         \\ \hline
Activator:        & LeakyReLU     & ReLU            \\
Output Activator: & Softmax       & Softmax         \\
Initialiser:      & he\_uniform   & -               \\
Optimiser:        & Adam          & SGD             \\
Momentum:         & N/A           & 0.9             \\
Early Stopping:   & 2             & 2               \\
Dropout:          & 0.2           & 0.25*3/0.2*2    \\
Learning Rate:    & 0.01          & 0.01            \\
Loss:             & CC            & CC              \\
Batch Norm:       & True          & True            \\
Hidden Layers:    & 4             & 5               \\
Nodes per Layer:  & 256/128/64/32 & 100/80/60/40/20 \\
Batch Size:       & 132           & 170             \\
Epochs:           & 20            & 20              \\ \hline
\end{tabular}
\end{table}

%Table \ref{tab:seq_nn} specifies the parameter values for a multi-layered feed-forward neural network model, consisting of one input layer (128 neurons) and one hidden layer (64 neurons) using the ReLu activator function. The output layer has a subsequent 7 neurons corresponding to the 7 different output classes, using a soft-max activation function to produce the class probabilities. See Appendix \ref{appx: MLP NN v1} for the full code.
%



%!TEX root =  ../Report.tex

\section{Analysis Of Results}
 \label{sec: Analysis Of Results}

% TODO Summarise the Results of each model here and talk about the best model here. 

In this section, the performance of the machine learning models on the test set is analysed and discussed. The models are evaluated using the metrics discussed previously, such as F1-Score, Area Under Curve (AUC), Precision, Recall and Accuracy. The best-performing models and algorithms are identified and interpreted. Finally, limitations and challenges faced in the analysis are discussed and addressed and areas of improvement for future work are suggested.

% \subsection{Performance on Test Set}

% Now begin the analysis of each model

\subsection{Comparison of Models}

% TODO Talk about the best models from each algorithm and then compare the results, features etc and conclude on one or more models that are the best for this classification problem.

\subsection{Model Interpretation}

% Discuss any insights gained from the analysis of the machine learning models, including which features were important for the models' decision-making process and how these features impacted the performance of the models.

\subsection{Limitations}

% Acknowledge any limitations of your analysis and potential sources of error. Explain how these limitations could impact the generalizability of your findings.

\subsection{Future Work}

% Suggest potential areas for future work that could build on your analysis and improve the performance of machine learning models in the specific problem you addressed and in the field more broadly.
%%!TEX root =  ../Report.tex
% %Here are some subsections so that they will appear on the contents
\subsection{Classifiers}

%Classification on the combined set of features was conducted against three classifier models, Random Forest (RF), XGBoost and K-Nearest Neighbour (KNN).

\subsubsection{Random Forest (RF)}
\label{sec:RF}

Basically existing research doesn't have any models or research using a random forest ensemble classifier, only decision trees in \parencite{9360747}

\begin{table}[h]
\centering
\caption{Parameters for Random Forest Classifier}
\label{tab:rf-params}
\begin{tabular}{|l|l|l|l|}
\hline
\textbf{Parameter} & \textbf{Description} & \textbf{Value}  \\ \hline
n\_estimators & The number of trees in the forest & 100\\
criterion & Function to measure the quality of a split. & gini\\
max\_depth & Maximum depth of the tree. &  None \\
min\_samples\_split & Minimum samples required to split an internal node. & 2 \\ 
min\_samples\_leaf & Minimum samples required to be at a leaf node. & 1 \\
max\_features & Maximum features to consider when splitting.  & auto \\
bootstrap & To bootstrap samples when constructing trees & True \\
class\_weight & Weights associated with classes & None  \\
random\_state & The random seed & 1234 \\ \hline
\end{tabular}
\end{table}

% \begin{table}[h]
% \centering
% \caption{RF Model Metrics}
% \label{tab:rf-metrics}
% \begin{tabular}{|l|l|l|l|l|l|l|}
% \hline
% \textbf{Model} & \textbf{Data Subset} & \textbf{Accuracy} & \textbf{Precision} & \textbf{Recall} & \textbf{F1} & \textbf{Time} \\ \hline
% Base & 80\% & 0.997 & 0.955 & 0.882 & 0.916 & 00:02:34:59 \\ \hline
% Base & 100\% & 0.997 & 0.997 & 0.997 & 0.997 & 00:00:30:60 \\ \hline
% \end{tabular}
% \end{table}

\begin{table}[h]
\centering
\caption{RF Model Metrics}
\label{tab:rf-metrics}
\begin{tabular}{|l|l|l|l|l|l|}
\hline
\textbf{Model} & \textbf{Data Subset} & \textbf{Accuracy} & \textbf{Precision} & \textbf{Recall} & \textbf{F1}  \\ \hline
Base & 80\% & 0.997 & 0.955 & 0.882 & 0.916  \\ \hline
Base & 100\% & 0.997 & 0.997 & 0.997 & 0.997 \\ \hline
\end{tabular}
\end{table}

\begin{table}[h]
\centering
\caption{RF Model v2}
\label{tab:rf-model2}
\begin{tabular}{|l|l|l|l|l|l|l|l|}
\hline
\textbf{Device} & \textbf{Model} & \textbf{Data Size} & \textbf{AUC} & \textbf{Precision} & \textbf{Recall} & \textbf{Accuracy} & \textbf{F1}  \\ \hline
GPU & Optimised & 100\% & 99.99 & 99.66 & 99.67 & 99.67 & 99.66 \\ \hline
\end{tabular}
\end{table}

\paragraph{Confusion Matrix}

\begin{figure}[H]
    \centering
    \includegraphics[width=0.85\textwidth]{Appendices/NN Confusion Matrix 3-04-23.png}
    \caption{RF Confusion Matrix}
    \label{fig:rf_confusion_matrix}
\end{figure}

\subsection{XGBoost}

 Table \ref{tab:xgb-scv-metrics} summarises the average metrics with 10 Fold Stratified Cross-Validation and Table \ref{tab:xgb-test-metrics} shows the metrics across the 30\% test set. Due to the numerous models created during experimentation, only the most notable models are included in the tables. The raw metrics for all models can be found in \ref{appx:XGBoost}. 

\begin{table}[h]
\centering
\caption{XGBoost S-CV Metrics}
\label{tab:xgb-scv-metrics}
\begin{tabular}{|l|l|l|l|l|l|l|l|}
\hline
\textbf{Model ID} & \textbf{Dataset} & \textbf{AUC} & \textbf{F1} & \textbf{Precision} & \textbf{Recall} & \textbf{Accuracy}  \\ \hline
6 & 100\% & 99.99 & 99.64 & 99.64 & 99.64 & 99.64 \\ \hline
8 & 100\% & 99.99 & 99.64 & 99.64 & 99.65 & 99.65 \\ \hline
10 & 100\% & 100.00 & 99.65 & 99.65 & 99.66 & 99.66 \\ \hline
11 & 100\% & 100.00 & 99.64 & 99.64 & 99.65 & 99.65 \\ \hline
\end{tabular}
\end{table}

\begin{table}[H]
\centering
\caption{XGBoost Test Metrics}
\label{tab:xgb-test-metrics}
\begin{tabular}{|l|l|l|l|l|l|l|l|}
\hline
\textbf{Model ID} & \textbf{Dataset} & \textbf{AUC} & \textbf{F1} & \textbf{Precision} & \textbf{Recall} & \textbf{Accuracy}  \\ \hline
0 & 60\% & 99.99 & 99.63 & 99.64 & 99.64 & 99.64 \\ \hline
1 & 80\% & 99.99 & 99.64 & 99.65 & 99.65 & 99.65 \\ \hline
2 & 80\% & 99.99 & 99.64 & 99.64 & 99.64 & 99.64 \\ \hline
3 & 80\% & 99.99 & 99.64 & 99.64 & 99.64 & 99.64 \\ \hline
5 & 80\% & 99.99 & 99.64 & 99.65 & 99.65 & 99.65 \\ \hline
6 & 100\% & 99.99 & 99.65 & 99.65 & 99.65 & 99.65 \\ \hline
8 & 100\% & 99.99 & 99.65 & 99.65 & 99.65 & 99.65 \\ \hline
10 & 100\% & 99.99 & 99.65 & 99.65 & 99.66 & 99.66 \\ \hline
11 & 100\% & 99.99 & 99.65 & 99.65 & 99.65 & 99.65 \\ \hline
\end{tabular}
\end{table}

\smallskip
Models 0, 1 and 6 were trained across varying levels of data sizes, but the models shared similar performance metrics. Model 0, trained on 60\% of the dataset slightly underperformed in classifying minority classes as seen in the F1, precision and recall scores from the classification report. Model 1 showed a similar pattern, but had a minor increase in correct classifications, especially for SSH. Model 6, being trained on more data, subsequently achieved a better result across most classes and can be seen in the classification report and confusion matrix. Precision and recall in the testing set were also increased. Although minor, the gradual improvement shows the positive impacts more data can have to help a model train and learn the patterns and complexities of the dataset, helping it to generalise well on unseen data.

\smallskip
In Model 3, 80\% of the dataset was used and early stopping of 10 rounds and regularisation of 0.1 were added to the model. However, analysing the metrics in detail models 1/2 share a similar performance. Both equally have high precision and recall and the weighted averages are very similar, the confusion matrices show some minor differences but were not significant to affect performance. It appears that the addition of early stopping and regularisation to model 3 did not significantly affect the performance of the model compared to the baseline. 

\medskip

\subsubsection*{GridSearchCV}

\smallskip
As mentioned previously, one instance of GridSearchCV ran successfully that tested a combination of the learning rate, number of estimators and the max depth. The test was cross-validated 5 times and took 28.27 hours to complete. Models 5 (80\%) and 8 (100\%) were created with these parameters. Compared to their baseline counterparts (4 and 6), the results are incomparable. Model 5 shares identical values for AUC and F1, with a 0.1 increase in Precision, Recall and Accuracy, this is also similar in Model 8, except AUC rounds to 100. To summarise, the models with default parameters (Models 4 and 6) and the models with the best-found parameters from GridSearchCV (Models 5 and 8) have similar performances across both datasets. Model 11 adopts the same parameters with the added inclusion of S-CV, early stopping and regularisation, however, there were no significant improvements with similar highly metrics and behaviour.

\subsubsection*{RandomisedSearchCV}

\medskip

A parameter grid was tested with RandomisedSearchCV and took a total of 41.81 hours to complete. Each parameter combination was subjected to 10-fold Stratified Cross Validation to ensure its performance was measured fairly. 

A new model was created with the best-found parameters, Model 10. The model performed very well, with an average AUC of 99.99 on the training data and 99.98 on the test data. The F1 score was 99.65 on the test set indicating the model has a high proportion of correct predictions, balancing both precision and recall well. The confusion matrix verifies this and shows the model to perform well for most classes, especially Normal, SDDP and Web Spoofing with almost perfect precision and recall. However, there are a few misclassifications for 'Botnet', 'Malware' and 'SSH'. The model struggled and occasionally misclassified Normal traffic as malicious, but performed well in the SQL Injection class, especially given the small number of samples. The Cross-validation and test set results are similar and indicate the model is not overfitting and generalising well to new data. 
Using the total number of instances of each class, the misclassification report can be calculated for each and is shown accordingly:

\begin{itemize}
	\item Botnet: 4208 / 17060 = {\color{red} 25\%}
	\item Malware: 7161 / 39476 = 18\%
	\item Normal: 6365 / 4572206 = 0.0013\%
	\item SQL: 89 / 789 = 11\%
	\item SSDP: 0 / 1649955 = {\color{mygreen} 0\%}
	\item SSH: 771 / 3565 = 21\%
	\item WebSpoof: 3046 / 121533 = 2.5\%
\end{itemize}

\subsection*{Feature Importance}

%% TODO Write section about the feature importances here.

In conclusion, the XGBoost classifier demonstrated a high level of performance across the classes for this classification problem. Despite the levels of imbalance, the models still held up relatively well. Techniques such as GridSearchCV and RandomisedSearchCV were used to fine-tune parameters; however, it became evident that despite this, the baseline parameter model could still deliver remarkably similar results. Future investigations may consider a bigger search grid with more focused tuning based on the specific characteristics of the classes or tasks.

% Further emphasis should be placed on the minority classes to strive towards a reduction in the number of false positives. The process of experimenting with and validating these models has underscored the need for a careful balance between precision and recall in the critical field of Intrusion Detection Systems (IDS).


% Refer to \ref{fig:xgb_optimised_fi}, \ref{fig:xgb_optimised_cm} and \ref{tab:optimised_xgboost} for metrics.
%\begin{figure}[H]
%\centering
%\caption{Optimised XGBoost Model FI}
%\includegraphics[width=\textwidth]{Appendices/Images/XGB/xgb_1010_fi.png}
%\label{fig:xgb_optimised_fi}
%\end{figure}
%
%\clearpage
%\begin{table}[hp]
%  \centering
%  \caption{Optimised XGBoost Classification Report}
%  \label{tab:optimised_xgboost}
%    \begin{tabular}{lcccc}
%    \toprule
%    Class & Precision & Recall & F1-Score & Support \\
%    \midrule
%    Botnet & 0.96 & {\color{red}\bfseries 0.75} & {\color{red}\bfseries 0.84} & 17060 \\
%    Malware & {\color{red}\bfseries 0.89} & 0.82 & 0.85 & 39476 \\
%    Normal & 1.00 & 1.00 & 1.00 & 4572206 \\
%    SQL Injection & 0.94 & 0.89 & 0.91 & 789 \\
%    SSDP & 1.00 & 1.00 & 1.00 & 1649955 \\
%    SSH & 0.92 & 0.78 & 0.85 & 3565 \\
%    Website Spoofing & 0.99 & 0.97 & 0.98 & 121533 \\
%    \midrule
%    Accuracy & & & 1.00 & 6404584 \\
%    Macro Avg & 0.96 & 0.89 & 0.92 & 6404584 \\
%    Weighted Avg & 1.00 & 1.00 & 1.00 & 6404584 \\
%    \bottomrule
%    \end{tabular}
%\end{table}
%
%\begin{figure}[H]
%\centering
%\caption{Optimised XGBoost Model CM}
%\includegraphics[width=0.9\textwidth]{Appendices/Images/XGB/xgb_1010_cm.png}
%\label{fig:xgb_optimised_cm}
%\end{figure}

\subsection{MLP}

In exploring Neural Networks, a series of MLP models were created with a varying number of parameters and was tested with different subsets of the dataset. Each model consisted of a different number of hidden layers and neurons, optimisers (Adam \& SGD), regularisation techniques (Dropout and Early Stopping), learning rates etc. The primary metrics for evaluating the models were AUC and F1, but the classification and confusion matrices were considered and used to form a detailed picture of each model's performance on the individual attack classes. Table \ref{tab:mlp-scv-metrics} and \ref{tab:mlp-test-metrics} show the S-CV and Test Set results for 8 MLP NN models. 

\begin{table}[h]
\centering
\caption{MLP S-CV Metrics}
\label{tab:mlp-scv-metrics}
\begin{tabular}{|l|l|l|l|l|l|l|l|}
\hline
\textbf{Model ID} & \textbf{Dataset} & \textbf{AUC} & \textbf{F1} & \textbf{Precision} & \textbf{Recall} & \textbf{Accuracy}  \\ \hline
4 & 100\% & 99.90 & 99.37 & 99.40 & 99.42 & 99.42 \\ \hline
5 & 100\% & {\color{red} 98.40} & {\color{red} 94.68} & {\color{red} 96.85} & {\color{red} 95.49} & {\color{red} 95.49} \\ \hline
6 & 100\% & 99.79 & 99.27 & 99.31 & 99.33 & 99.33 \\ \hline
7 & 100\% & 99.72 & 99.23 & 99.25 & 99.31 & 99.31 \\ \hline

\end{tabular}
\end{table}

\begin{table}[H]
\centering
\caption{MLP Test Metrics}
\label{tab:mlp-test-metrics}
\begin{tabular}{|l|l|l|l|l|l|l|l|}
\hline
\textbf{Model ID} & \textbf{Dataset} & \textbf{AUC} & \textbf{F1} & \textbf{Precision} & \textbf{Recall} & \textbf{Accuracy}  \\ \hline
0 & 60\% & 99.99 & 99.36 & 99.38 & 99.41 & 99.41 \\ \hline
1 & 60\% & 99.99 & 99.34 & 99.36 & 99.38 & 99.38 \\ \hline
2 & 80\% & 99.86 & 99.42 & 99.44 & 99.39 & 99.44 \\ \hline
3 & 80\% & 99.98 & 99.39 & 99.44 & 99.44 & 99.44 \\ \hline
4 & 100\% & 99.94 & 99.42 & 99.44 & 99.46 & 99.46 \\ \hline
5 & 100\% & 99.88 & 99.36 & 99.37 & 99.40 & 99.40 \\ \hline
6 & 100\% & 99.80 & 99.28 & 99.40 & 99.42 & 99.42 \\ \hline
7 & 100\% & 99.84 & 99.29 & 99.33 & 99.35 & 99.35 \\ \hline
\end{tabular}
\end{table}

\subsubsection*{Data Subsets}

Examining the results across the different data subsets, both 60\% and 80\% models showed high precision and recall. Class-specific performances for minority classes were consistently low. Notable, in model 3 on the 80\% subset there was an increase in recall within the SQL Injection class. The models exhibited high overall performance but struggled with frequent misclassifications which suggest more data is required for the model to correctly identify the specific classes.

\subsubsection*{LeakyReLU}

Models 5 and 6 differ in the selection of the activation function. (ReLU in model 5 and LeakyReLU in model 6). Upon initial examination, there are differences in test metrics, but larger differences appear when looking at class-specific performances. 
Whilst the precision was increased in some classes such as Botnet and SSH, recall suffered substantially such as 0.13 for SQL Injection, indicating the model was able to reduce some false positives at the high cost of failing to identify the true positives. 

\subsubsection*{Previous Works}
The works by \textcite{s22155633} similarly used an MLP model consisting of four hidden layers, these specifications were adopted in Model 7 to provide context. The model displayed an AUC of 99.84, F1 of 99.28, Recall of 99.35 and Accuracy of 99.35 on the test set. However, the precision, recall and F1 for SQL Injection are substantially lower at 0.02 and 0.05 compared to other classes. The model fails to identify this class accurately, similarly, Botnet and Malware also saw a drop in performance. The Confusion matrix further affirms this observation with a large number of predictions from those classes being misclassified as Normal traffic. This further emphasises the importance of tuning parameters and settings that are specific to the problem at hand. 

%\begin{table}[htbp]
%  \centering
%  \caption{MLP v1 Classification Report}
%  \label{tab:mlp_v1_class_report}
%    \begin{tabular}{lcccc}
%    \toprule
%    Class & Precision & Recall & F1-Score & Support \\
%    \midrule
%    Botnet & 0.65 & 0.53 & 0.58 & 8530 \\
%    Malware & 0.83 & 0.68 & 0.75 & 19738 \\
%    Normal & 0.99 & 1.00 & 0.99 & 2286103 \\
%    SQL Injection & 0.98 & {\color{red}\bfseries 0.23} & 0.37 & 395 \\
%    SSDP & 1.00 & 1.00 & 1.00 & 824978 \\
%    SSH & {\color{red}\bfseries 0.58} & 0.38 & {\color{red}\bfseries 0.46} & 1782 \\
%    Website Spoofing & 0.92 & 0.92 & 0.92 & 60766 \\
%    \midrule
%    Accuracy & & & 0.99 & 3202292 \\
%    Macro Avg & 0.85 & 0.68 & 0.72 & 3202292 \\
%    Weighted Avg & 0.99 & 0.99 & 0.99 & 3202292 \\
%    \bottomrule
%    \end{tabular}%
%\end{table}%
%
%After tuning, we proceeded to create an additional model with additional layers 
%
%\begin{table}[hp]
%\captionsetup{justification=centering} 
%\centering
%\caption{MLP v2 Specifications}
%\begin{tabular}{ll}
%\hline
%\textbf{Parameter} & \textbf{Value} \\ \hline
%Activator & Relu \\
%Output Activator & Softmax \\
%Initialiser & he\_uniform \\
%Optimiser & Adam \\
%Momentum & N/A \\
%Early Stopping & N/A \\
%Dropout & 0.2 \\
%Learning Rate & 0.001 \\
%Loss & Categorical Crossentropy \\
%Batch Norm & Yes \\
%Hidden Layers & 3 \\
%Nodes per Layer & 128, 64, 32 \\
%Batch Size & 180 \\
%Epochs & 15 \\ \hline
%\end{tabular}
%\label{tab:mlp_v2}
%\end{table}

\subsubsection*{Summary}

The Multi-Layer Perceptron in TensorFlow can provide a vast array of parameters and options, leading to an endless number of combinations to be tailored and optimised. Finding the 'best' model in the classification problem is a difficult non-trivial task. 

With challenges and limitations in the hardware and time, the approach for these experiments consisted of creating a sequence of models and comparing the performance metrics to previous models and the overall domain knowledge and task.

Experimentation stopped after a vast number of models were tested and reached diminishing returns. Whilst this approach does not guarantee the 'best' MLP configuration, it provides a practical and effective method that strikes a balance between complexity and constraints. Further work can be investigated into utilising parameter searching such as GridSearchCV to automate the process.

\subsection{Comparison of Models}

The purpose of this work serves to provide recommendations for developing a wireless network intrusion detection system, however, evaluating the performances of the models poses a challenge when determining the 'best' model for each algorithm. Challenges arise when multiple metrics and performance indicators are to be compared. The analysis focused on achieving a balance between avoiding false positives (instances where normal traffic is marked as malicious) and false negatives (instances of malicious traffic marked as normal). The aim was to identify a model that was capable of distinguishing between the 6 attack classes and avoiding as many false negatives and positives as possible. The following sections compare the three 'best' identified models from the following models: Random Forest, XGBoost and Multi-Layer Perceptron (MLP).

\subsubsection*{Random Forest}

During experimentation, several Random Forest models were trained, and attempts were made to search through a series of parameters, using the evaluation metrics defined previously, Model ID 1 displayed strong consistent performance during CV and testing. It achieved an AUC of 99.99, F1 of 99.66, Precision of 99.66, Recall of 99.67 and Accuracy of 99.67 on the test set, indicating that it was able to correctly classify the six attack classes with a high degree of accuracy. The model's confusion matrix shows a good balance between FP and FNs on majority classes, whilst it did struggle slightly on Botnet and SSH, the total number of misclassification remained relatively low. \ref{tab:rf_class_report}, \ref{fig:rf_model1_cm} and \ref{fig:rf_model1_fi} show the Classification, Confusion Matrix and Feature Importances for the model.

\begin{table}[htbp]
  \centering
  \caption{RF Model 1 Classification Report}
  \label{tab:rf_class_report}
    \begin{tabular}{lcccc}
    \toprule
    Class & Precision & Recall & F1-Score & Support \\
    \midrule
    Botnet & 0.95 & {\color{red}\bfseries 0.77} & {\color{red}\bfseries 0.85} & 17060 \\
    Malware & {\color{red}\bfseries 0.89} & 0.82 & 0.86 & 39476 \\
    Normal & 1.00 & 1.00 & 1.00 & 4572206 \\
    SQL Injection & 0.93 & 0.86 & 0.89 & 789 \\
    SSDP & 1.00 & 1.00 & 1.00 & 1649955 \\
    SSH & 0.94 & 0.79 & 0.86 & 3565 \\
    Website Spoofing & 0.99 & 0.98 & 0.98 & 121533 \\
    \midrule
    Accuracy & & & 1.00 & 6404584 \\
    Macro Avg & 0.96 & 0.89 & 0.92 & 6404584 \\
    Weighted Avg & 1.00 & 1.00 & 1.00 & 6404584 \\
    \bottomrule
    \end{tabular}
\end{table}

\begin{figure}[H]
    \centering
	\includegraphics[width=0.8\textwidth]{Appendices/Images/RF/Model1/RF_Model1_CM.png}
	\caption{RF Model 1 CM}
  	\label{fig:rf_model1_cm}
\end{figure}

\begin{figure}[H]
    \centering
	\includegraphics[width=\textwidth]{Appendices/Images/RF/Model1/RF_Model1_FI.png}
	\caption{RF Model 1 FI}
  	\label{fig:rf_model1_fi}
\end{figure}


\subsubsection*{XGBoost}

Comparing all 11 models trained on the XGBoost Classifier, Model 10 achieved superiority with an AUC of 99.99 (rounding to 100), F1 of 99.65, Precision of 99.65, Recall of 99.66 and Accuracy of 99.66 across the test set and similar during Cross Validated training.  

%% TODO XGBOOST Best Model

\begin{table}[htbp]
  \centering
  \caption{XGBoost Best Model Classification Report}
  \label{tab:xgb_class_report}
    \begin{tabular}{lcccc}
    \toprule
    Class & Precision & Recall & F1-Score & Support \\
    \midrule
    Botnet & 0.96 & 0.75 & {\color{red}\bfseries 0.84} & 17060 \\
    Malware & {\color{red}\bfseries 0.89} & 0.82 & 0.85 & 39476 \\
    Normal & 1.00 & 1.00 & 1.00 & 4572206 \\
    SQL Injection & 0.94 & 0.89 & {\color{red}\bfseries 0.91} & 789 \\
    SSDP & 1.00 & 1.00 & 1.00 & 1649955 \\
    SSH & 0.92 & {\color{red}\bfseries 0.78} & 0.85 & 3565 \\
    Website Spoofing & 0.99 & 0.97 & 0.98 & 121533 \\
    \midrule
    Accuracy & & & 0.99 & 6404584 \\
    Macro Avg & 0.83 & 0.73 & 0.80 & 6404584 \\
    Weighted Avg & 0.99 & 0.99 & 0.99 & 6404584 \\
    \bottomrule
    \end{tabular}
\end{table}

\begin{figure}[H]
	\centering
	\includegraphics[width=0.8\textwidth]{Appendices/Images/XGB/Model10/XGB_Model10_CM.png}
	\caption{XGBoost Model 10 CM}
  	\label{fig:xgb_model10_cm}
\end{figure}

\begin{figure}[H]
	\centering
	\includegraphics[width=\textwidth]{Appendices/Images/XGB/Model10/XGB_Model10_FI.png}
	\caption{XGBoost Model10 Feature Importance}
  	\label{fig:xgb_model10_fi}
\end{figure}


% MLP
\subsubsection*{MLP}

Out of all models trained, model 4 showed strong performance overall with a good proportion of instances from classes Botnet, Malware, Normal and SDDP being accurately classified. While the model struggles with SQL Injection and SSH with poor recall and F1 scores, this was expected given the imbalance in the dataset. Metrics were consistent across S-CV and the test set, indicating that the MLP model was not overfitting the training data and generalising well on new unseen data. Table \ref{tab:mlp_class_report} and \ref{fig:mlp_model4_cm} shows the Classification Report and Confusion Matrix for the model. 

\begin{table}[htbp]
  \centering
  \caption{MLP Best Model Classification Report}
  \label{tab:mlp_class_report}
    \begin{tabular}{lcccc}
    \toprule
    Class & Precision & Recall & F1-Score & Support \\
    \midrule
    Botnet & 0.94 & 0.61 & 0.74 & 17060 \\
    Malware & 0.89 & 0.72 & 0.80 & 39476 \\
    Normal & 0.99 & 1.00 & 1.00 & 4572206 \\
    SQL Injection & 0.99 & {\color{red}\bfseries 0.37} & {\color{red}\bfseries 0.54} & 789 \\
    SSDP & 1.00 & 1.00 & 1.00 & 1649955 \\
    SSH & {\color{red}\bfseries 0.83} & 0.48 & 0.60 & 3565 \\
    Website Spoofing & 1.00 & 0.92 & 0.95 & 121533 \\
    \midrule
    Accuracy & & & 0.99 & 6404584 \\
    Macro Avg & 0.83 & 0.73 & 0.80 & 6404584 \\
    Weighted Avg & 0.99 & 0.99 & 0.99 & 6404584 \\
    \bottomrule
    \end{tabular}
\end{table}

\begin{figure}[H]
    \centering
	\includegraphics[width=0.8\textwidth]{Appendices/Images/MLP/MLP_Model4_CM.png}
	\caption{MLP Model 4 CM}
  	\label{fig:mlp_model4_cm}
\end{figure}

\subsection{Limitations}

The experiments faced several limitations that should be considered in the interpretation of results and impact findings in this dissertation. 

Firstly, the constraints of time and computational resources hindered the ability to construct and test an exhaustive list of models and the frequent occurrence of hardware and OS crashes negatively disrupted model training, leading to a loss in progress. 
GridSearchCV is a searching method used to find optimal parameters, it was not utilised to the full extent due to similar constraints, which meant models were not tuned exhaustively, possibly leading to lower performance.

Lastly, the iterative and exploratory approach taken for model selection and tuning could be a source of error. This may have been susceptible to undue bias or randomness, potentially leading to under-explored models. 

These limitations prove that although the results are high, consideration should be exercised with caution with evaluating their use for a proposed IDS. Additional research is needed to validate these models and further work is suggested to investigate other, more complex and tuned models.

\subsection{Recommendations}

Comparing all three machine learning algorithms, experiments from this study show that the Random Forest model ran with default parameters displayed slightly better results compared to the other two. The XGBoost model also performed very well with similar results but had a small increase in false positives for Botnet, Malware and SSH. The MLP model, whilst still displaying high levels of performance, appeared to struggle with the classification of minority classes. The Confusion matrix contains a high number of false negatives for Botnet, Malware and SSH, suggesting it was unable to accurately predict these classes.

As mentioned, identifying the most optimal solution for a proposed Wireless Network Intrusion Detection System is a highly subjective challenge. The final decision hinges on the performance of the models within the author's environment and ultimately depends on the particular needs and focuses in the production environment. 

% !TEX root =  ../Report.tex

\section{Conclusion}
\label{sec: Conclusion}

% \lipsum[7]

\subsection{Summary Of Findings}

Summarize the main findings of your research, including the performance of your machine learning models, insights gained from the analysis, and any limitations or areas for future work.

\subsection{Implications}

 Discuss the implications of your research for the field of machine learning and for the specific problem you addressed. This could include discussing how your research advances the state-of-the-art, how it could be applied in practice, or how it could inform future research.
 
\subsection{Contributions}

Discuss the contributions of your research to the field of machine learning. This could include discussing how your research fills a gap in the literature, how it provides new insights into the problem you addressed, or how it advances the methodology used in the field.

\subsection{Limitations}

Discuss the limitations of your research and potential sources of error. Explain how these limitations could impact the generalizability of your findings and suggest ways to address these limitations in future research.

\subsection{Future Work}

Discuss potential future work that could be done to build on your research and improve the performance of machine learning models in the specific problem you addressed and in the field more broadly.

\include{Chapter7/Chapter7}
%!TEX root =  ../Report.tex

%\section{Conclusions}
%\label{sec:Chapter 8}

%\lipsum[9]



\include{Chapter9/Chapter9}
%Keep adding folders as to your desires

% \bibliographystyle{abbrv}
%\nocite{*}
\printbibliography[heading=bibintoc,title={References}]

\begin{appendices}
\addcontentsline{toc}{section}{Appendices}


%\section{The Flux Sketch}
%\includegraphics[width=\textwidth]{Report/Appendices/flux_sketch.jpg}
%\label{appx: Flux Sketch}

\section{Ethical Approval}
\begin{lstlisting}[language={}]
Dear 2054584,

Warwick ID Number: 2054584 

This is to confirm that your Supervisor's Delegated Approval form has been received by the WMG Full-time Student Office, Detecting IEEE 802.11 Attacks using Machine Learning does NOT require ethical approval.

You are reminded that you must now adhere to the answers and detail given in the completed WMG SDA ethical approval form (and associated documentation) within your research project. If anything changes in your research such that any of your answers change, then you must contact us to check if you need to reapply for or update your ethical approval before you proceed.

If your data collection strategy, including the detail of any interview/ survey questions that you drafted changes substantially prior to or during data collection, then you must reapply for ethical approval before your changes are implemented. 

When you submit your project please write N/A against the ethical approval field in the submission pro-forma and include a copy of this email in the appendices of your project.

Kind regards

Jade Barrett
\end{lstlisting}

\newpage
\section{Dataset Manipulation}

\subsection{CSV Combiner Script}
\label{appx: CSV Combiner Script}
\begin{lstlisting}[language=bash, literate={-}{-}1]
#!/bin/bash

# Input Directory
input_dir="../Malware"

cd "$input_dir"

# Set the output file name
output_file="combined.csv"

# Check if the output file already exists and delete it
if [ -f "$output_file" ]; then
  rm "$output_file"
fi

# Print a status message
echo "Combining files..."

# Loop through all the files that match the pattern reduced_*.csv
for file in $(ls *.csv | sort -V)
do
  # Check if the file exists
  if [ -f "$file" ]; then
    # Print a status message
    echo "Combining $file..."

# If this is the first file, copy the header to the output file
    if [ ! -f "$output_file" ]; then
      head -n 1 "$file" > "$output_file"
    fi

    # Append all the rows except the header to the output file
    tail -n +2 "$file" >> "$output_file"
  fi
done

# Print a status message
echo "Done."
\end{lstlisting}

\newpage
\subsection{Feature Extraction \& Reduction}
\label{appx: Feature Extraction}

\begin{lstlisting}[language=Python]
# Define the columns to extract
cols_to_use = [
	'frame.len','radiotap.dbm_antsignal','radiotap.length', 
	'wlan.duration','wlan_radio.duration','wlan_radio.signal_dbm', 
	'radiotap.present.tsft','wlan.fc.type','wlan.fc.subtype', 
	'wlan.fc.ds','wlan.fc.frag','wlan.fc.moredata',
	'wlan.fc.protected','wlan.fc.pwrmgt','wlan.fc.retry',
	'wlan_radio.phy','udp.length','ip.ttl',
	'arp','arp.proto.type','arp.hw.size',
	'arp.proto.size','arp.hw.type','arp.opcode',
  'tcp.analysis','tcp.analysis.retransmission','tcp.option_len',
  'tcp.checksum.status','tcp.flags.ack','tcp.flags.fin',
  'tcp.flags.push','tcp.flags.reset','tcp.flags.syn',
  'dns','dns.count.queries','dns.count.answers',
  'dns.resp.len','dns.resp.ttl','http.request.method',
  'http.response.code','http.content_type','ssh.message_code',
  'ssh.packet_length','nbns','nbss.length',
  'nbss.type','ldap','smb2.cmd',
  'smb.flags.response','smb.access.generic_read',
  'smb.access.generic_write','smb.access.generic_execute',
  'Label']

# Define the chunk size you want to read in each iteration
batch_size = 1000000

# Initialize an empty dataframe to hold the combined results
combined_df = pd.DataFrame()

# Iterate through the file in batches
for chunk in pd.read_csv('botnet_combined.csv', chunksize=batch_size, usecols=cols_to_use, low_memory=False):
    
    # Combine the processed chunk with previous chunks
    combined_df = pd.concat([combined_df, chunk])
\end{lstlisting}

\begin{lstlisting}[language=Python,linewidth=\textwidth]
# Drop all missing rows that contain only nan values
combined_df = combined_df.dropna(how='all')

# Drop all rows with missing values in Label Column
combined_df = combined_df.dropna(subset=['Label'])

# Fill NAs with zeros
# Change nan values to 0
combined_df = combined_df.fillna(0)
\end{lstlisting}

\newpage
\begin{lstlisting}
# Duplicate the dataframe
df = combined_df.copy()

# Regex to keep only the first value e.g 
# -100-100-10 becomes -100,   123-456-1 becomes 123, -10-2 becomes-10, 81-63-63 becomes 81
def seperated_values(x):
    x = str(x)
    match = re.match(r'^(-?\d+).*$', x)
    if match:
        return match.group(1)
    else:
        return x

# Go through all columns and change seperate values into just one value
for column in df.columns:
    df[column] = df[column].apply(seperated_values)
    print('Processing', column)
print('Done')

# Find Rows that contain values such as Oct-26, Oct-18, Feb-10 etc.. as these appear to be invalid and we will drop these rows.
regex = r"\b(?:\d{2}|(?:Jan|Feb|Mar|Apr|May|Jun|Jul|Aug|Sep|Oct|Nov|Dec))-(?:\d{2}|(?:Jan|Feb|Mar|Apr|May|Jun|Jul|Aug|Sep|Oct|Nov|Dec))\b"

# Use str.match method to apply the regex pattern to the column
mask = df['tcp.option_len'].astype(str).str.match(regex).fillna(False)
df = df[~mask]

mask = df['dns.resp.ttl'].astype(str).str.match(regex).fillna(False)
df = df[~mask]

mask = df['ip.ttl'].astype(str).str.match(regex).fillna(False)
df = df[~mask]

mask = df['smb2.cmd'].astype(str).str.match(regex).fillna(False)
df = df[~mask]

df.to_csv('Botnet_Reduced.csv', index=False)    

\end{lstlisting}

\newpage

\section{Conda Environments}
\label{appx: Conda_Env}

\subsection{Neural Networks - Apple Silicon}

\begin{lstlisting}
conda create -n nn-env python=3.9
conda activate nn-env
conda install -c apple tensorflow-deps
conda install -c conda-forge -y pandas jupyter
pip install tensorflow-macos==2.10
pip install numpy, matplotlib, scikit-learn, scipy, seaborn
\end{lstlisting}

\subsection{Classifiers}

\begin{lstlisting}
# Conda environment used for Random Forest, XGBoost and K-NN.

conda create -n ml-env python=3.9
conda activate ml-env
conda install -c conda-forge -y pandas jupyter
pip install numpy, matplotlib, scikit-learn, scipy, seaborn, xgboost
\end{lstlisting}

\newpage

\section{Data Preprocessing}
\label{appx:Data Processing}

\subsection{MinMax Scaling}
\label{appx:Scaling}

\begin{lstlisting}
# Define the scaler
scaler = MinMaxScaler()

# Fit the scaler to the following columns we define
scale_cols = [
        'frame.len',
        'radiotap.dbm_antsignal', 
        'radiotap.length', 
        'wlan.duration', 
        'wlan_radio.duration', 
        'wlan_radio.signal_dbm',
        'ip.ttl', 
        'udp.length', 
        'nbss.length',
        'dns.count.answers', 
        'dns.count.queries',
        'dns.resp.ttl',
        'ssh.packet_length']
        
# Fit the X_train and X_test
X_train[scale_cols] = scaler.fit_transform(X_train[scale_cols])
X_test[scale_cols] = scaler.transform(X_test[scale_cols])
\end{lstlisting}

\subsection{OHE Encoding}
\label{appx:OHE Encoding}
\begin{lstlisting}
cols_to_encode = [col for col in X_train.columns if col not in scale_cols]
X_all = pd.concat([X_train, X_test], axis=0)

X_all_ohe = pd.get_dummies(X_all, columns=cols_to_encode, drop_first=True, dtype=np.uint8)

# split back into train and test sets
X_train_ohe = X_all_ohe[:len(X_train)]
X_test_ohe = X_all_ohe[len(X_train):]
\end{lstlisting}

\newpage
\subsection{Label Encoding}
\label{appx:Label Encoding}
\begin{lstlisting}
    # Use Label Encoder to encode the target variable
le = LabelEncoder()

label_encoder = le.fit(y_train)
y_train_encoded = label_encoder.transform(y_train)
\end{lstlisting}

\subsection{Loading Dataset}
\label{appx:Loading Dataset}
\begin{lstlisting}
chunk_size = 1000000
dtype_opt = {
    'frame.len': 'int64',
    'radiotap.dbm_antsignal': 'int64',
    'radiotap.length': 'int64',
    'radiotap.present.tsft': 'int64',
    'wlan.duration': 'int64',
    'wlan.fc.ds': 'int64',
    'wlan.fc.frag': 'int64',
    'wlan.fc.moredata': 'int64',
    'wlan.fc.protected': 'int64',
    'wlan.fc.pwrmgt': 'int64',
    'wlan.fc.type': 'int64',
    'wlan.fc.retry': 'int64',
    'wlan.fc.subtype': 'int64',
    'wlan_radio.duration': 'int64',
    'wlan_radio.signal_dbm': 'int64',
    'wlan_radio.phy': 'int64',
    'arp': 'object',
    'arp.hw.type': 'object',
    'arp.proto.type': 'int64',
    'arp.hw.size': 'int64',
    'arp.proto.size': 'int64',
    'arp.opcode': 'int64',
    'ip.ttl': 'int64',
    'tcp.analysis': 'int64',
    'tcp.analysis.retransmission': 'int64',
    'tcp.checksum.status': 'int64',
    'tcp.flags.syn': 'int64',
    'tcp.flags.ack': 'int64',
    'tcp.flags.fin': 'int64',
    'tcp.flags.push': 'int64',
    'tcp.flags.reset': 'int64',
    'tcp.option_len': 'int64',
    'udp.length': 'int64',
    'nbns': 'object',
    'nbss.length': 'int64',
    'ldap': 'object',
    'smb2.cmd': 'int64',
    'dns': 'object',
    'dns.count.answers': 'int64',
    'dns.count.queries': 'int64',
    'dns.resp.ttl': 'int64',
    'http.content_type': 'object',
    'http.request.method': 'object',
    'http.response.code': 'int64',
    'ssh.message_code': 'int64',
    'ssh.packet_length': 'int64'
}

# Read the data
print('Reading X...')
X = pd.DataFrame()
for chunk in pd.read_csv('X.csv', chunksize=chunk_size, usecols=dtype_opt.keys(), dtype=dtype_opt, low_memory=False):
    X = pd.concat([X, chunk])

print('Reading y...')
y = pd.DataFrame()
for chunk in pd.read_csv('y.csv', chunksize=chunk_size, usecols=['Label'], dtype='object', low_memory=False):
   y = pd.concat([y, chunk])

# Split the data into training and testing sets
print('Splitting the data...')
X_train, X_test, y_train, y_test = train_test_split(X, y, test_size=0.30, random_state=1234, stratify=y)
\end{lstlisting}

\newpage
\section{Classifiers}
\label{appx: Classifiers}

\subsection{K-Nearest Neighbor (KNN)}
\begin{lstlisting}
# Use KNN
from sklearn.neighbors import KNeighborsClassifier

k=5

# Create KNN classifier
knn = KNeighborsClassifier(n_neighbors=k, n_jobs=-1)

# Fit the model
knn.fit(X_train_ohe, y_train_encoded)

# predict the test set
y_knn_pred = knn.predict(X_test_ohe)

from sklearn.metrics import classification_report, roc_auc_score

# Get the classification report
report = classification_report(y_test_encoded, y_knn_pred)

print('Classification Report:\n', report)

# Get the all the metrics for the multi class classification

print('Accuracy: ', accuracy_score(y_test_encoded, y_knn_pred))
print('Precision: ', precision_score(y_test_encoded, y_knn_pred, average='macro'))
print('Recall: ', recall_score(y_test_encoded, y_knn_pred, average='macro'))
print('F1 Score: ', f1_score(y_test_encoded, y_knn_pred, average='macro'))

# Get the confusion matrix for multi-class and plot it
confusion = confusion_matrix(y_test, y_rf_pred)
print('Confusion Matrix\n')
print(confusion)

# Plot the confusion matrix for multi-class classification using seaborn
labels = ['Normal', 'SSDP', 'Website Spoofing', 'Malware', 'Botnet', 'SSH', 'SQL Injection']

plt.figure(figsize=(8, 8))
sns.heatmap(confusion, annot=True, fmt='d', cmap='Blues', xticklabels=labels, yticklabels=labels)
plt.title('Confusion Matrix')
plt.xlabel('Predicted')
plt.ylabel('Actual')
plt.show()

plt.figure(figsize=(10, 10))
feat_importances = pd.Series(rf.feature_importances_, index=X_train_ohe.columns)
feat_importances.nlargest(20).plot(kind='barh')
plt.show()
\end{lstlisting}

\newpage

\subsection{Random Forest}
\label{appx:Random Forest}

\subsubsection{RF Model ID 0 - Raw Metrics}
\begin{lstlisting}[escapechar=!]
!\textbf{S-CV Results}!
Mean AUC = 99.99
Mean F1 = 99.66
Mean Precision = 99.66
Mean Recall = 99.67
Mean Accuracy = 99.67
Training Time: 7795 seconds

!\textbf{Final Test Results}!
Weighted AUC: 0.9999070506312879
Weighted F1: 0.996638797834701
Weighted Precision: 0.9966379719195173
Weighted Recall: 0.9967196932696956
Accuracy: 0.9967196932696956

!\textbf{Classification Report}!
              precision    recall  f1-score   support

      Botnet       0.95      0.77      0.85     17060
      Malware      0.89      0.82      0.86     39476
      Normal       1.00      1.00      1.00     457220
      SQL          0.93      0.86      0.89     789
      SDDP         1.00      1.00      1.00     1649955
      SSH          0.94      0.79      0.86     3565
      Spoofing     0.99      0.98      0.98     121533

    accuracy                           1.00   6404584
   macro avg       0.96      0.89      0.92   6404584
weighted avg       1.00      1.00      1.00   6404584

!\textbf{Confusion Matrix}!
[[  13082      17    3957       0       0       2       2]
 [     17   32454    6994       0       0      10       1]
 [    649    3821 4565771      51       2     161    1751]
 [      0       0     113     676       0       0       0]
 [      0       0       2       0 1649953       0       0]
 [      5      20     707       0       0    2833       0]
 [      4       0    2723       0       0       0  118806]]
\end{lstlisting}

\begin{center}
	\includegraphics[scale=0.8]{Appendices/Images/RF/rf_stock_cm.png}
\end{center}


\includegraphics[width=\textwidth]{Appendices/Images/RF/rf_stock_feature_imp.png}

\newpage
\subsubsection{RF Model ID 1 - Raw Metrics}
\begin{lstlisting}[escapechar=!]
!\textbf{S-CV Results}!
Mean AUC = 99.99
Mean F1 = 99.66
Mean Precision = 99.66
Mean Recall = 99.67
Mean Accuracy = 99.67
Training Time 7794.549654006958 seconds

!\textbf{Final Test Results}!
Test AUC: 0.9999070506312879
Weighted Test F1: 0.996638797834701
Weighted Test Precision: 0.9966379719195173
Weighted Test Recall: 0.9967196932696956
Test Accuracy: 0.9967196932696956

!\textbf{Classification Report}!
			  precision    recall  f1-score   support

      Botnet       0.95      0.77      0.85     17060
      Malware      0.89      0.82      0.86     39476
      Normal       1.00      1.00      1.00   4572206
         SQL       0.93      0.86      0.89       789
        SSDP       1.00      1.00      1.00   1649955
         SSH       0.94      0.79      0.86      3565
WebsiteSpoof       0.99      0.98      0.98    121533

    accuracy                           1.00   6404584
   macro avg       0.96      0.89      0.92   6404584
weighted avg       1.00      1.00      1.00   6404584
    
!\textbf{Confusion Matrix}!    
[[  13082      17    3957       0       0       2       2]
 [     17   32454    6994       0       0      10       1]
 [    649    3821 4565771      51       2     161    1751]
 [      0       0     113     676       0       0       0]
 [      0       0       2       0 1649953       0       0]
 [      5      20     707       0       0    2833       0]
 [      4       0    2723       0       0       0  118806]]
\end{lstlisting}


\newpage
\subsubsection{RF Model ID 1 - Raw Metrics}
\begin{lstlisting}[escapechar=!]
!\textbf{S-CV Results}!
Mean AUC = 99.99
Mean F1 = 99.66
Mean Precision = 99.66
Mean Recall = 99.67
Mean Accuracy = 99.67
Training Time 7794.549654006958 seconds

!\textbf{Final Test Results}!
Test AUC: 0.9999070506312879
Weighted Test F1: 0.996638797834701
Weighted Test Precision: 0.9966379719195173
Weighted Test Recall: 0.9967196932696956
Test Accuracy: 0.9967196932696956

!\textbf{Classification Report}!
			  precision    recall  f1-score   support

      Botnet       0.95      0.77      0.85     17060
      Malware      0.89      0.82      0.86     39476
      Normal       1.00      1.00      1.00   4572206
         SQL       0.93      0.86      0.89       789
        SSDP       1.00      1.00      1.00   1649955
         SSH       0.94      0.79      0.86      3565
WebsiteSpoof       0.99      0.98      0.98    121533

    accuracy                           1.00   6404584
   macro avg       0.96      0.89      0.92   6404584
weighted avg       1.00      1.00      1.00   6404584
    
!\textbf{Confusion Matrix}!    
[[  13082      17    3957       0       0       2       2]
 [     17   32454    6994       0       0      10       1]
 [    649    3821 4565771      51       2     161    1751]
 [      0       0     113     676       0       0       0]
 [      0       0       2       0 1649953       0       0]
 [      5      20     707       0       0    2833       0]
 [      4       0    2723       0       0       0  118806]]
 
\end{lstlisting}

\newpage
\subsubsection{RF Model ID 2 - Raw Metrics}
\begin{lstlisting}[escapechar=!]
!\textbf{Final Test Results}!
Test AUC:  0.9999070506308619
Weighted Test Precision:  0.9966379719195173
Weighted Test Recall:  0.9967196932696956
Weighted Test F1:  0.996638797834701
Test Accuracy:  0.9967196932696956

!\textbf{Classification Report}!
				  precision    recall  f1-score   support

          Botnet       0.95      0.77      0.85     17060
         Malware       0.89      0.82      0.86     39476
          Normal       1.00      1.00      1.00   4572206
   SQL_Injection       0.93      0.86      0.89       789
            SSDP       1.00      1.00      1.00   1649955
             SSH       0.94      0.79      0.86      3565
Website_spoofing       0.99      0.98      0.98    121533

        accuracy                           1.00   6404584
       macro avg       0.96      0.89      0.92   6404584
    weighted avg       1.00      1.00      1.00   6404584
    
!\textbf{Confusion Matrix}!    
[[  13082      17    3957       0       0       2       2]
 [     17   32454    6994       0       0      10       1]
 [    649    3821 4565771      51       2     161    1751]
 [      0       0     113     676       0       0       0]
 [      0       0       2       0 1649953       0       0]
 [      5      20     707       0       0    2833       0]
 [      4       0    2723       0       0       0  118806]]
\end{lstlisting}

\newpage
\subsubsection{RF Model ID 3 - Raw Metrics}
\begin{lstlisting}[escapechar=!]
!\textbf{S-CV Results}!
Mean AUC = 0.9999
Mean F1 = 0.9966
Mean Precision = 0.9966
Mean Recall = 0.9967
Mean Accuracy = 0.9967
Training Time 56042.87267756462  seconds

!\textbf{Final Test Results}!
Weighted AUC:  0.9999070506308619
Weighted F1:  0.996638797834701
Weighted Precision:  0.9966379719195173
Weighted Recall:  0.9967196932696956
Accuracy:  0.9967196932696956

!\textbf{Classification Report}!
				  precision    recall  f1-score   support

          Botnet       0.95      0.77      0.85     17060
         Malware       0.89      0.82      0.86     39476
          Normal       1.00      1.00      1.00   4572206
   SQL_Injection       0.93      0.86      0.89       789
            SSDP       1.00      1.00      1.00   1649955
             SSH       0.94      0.79      0.86      3565
Website_spoofing       0.99      0.98      0.98    121533

        accuracy                           1.00   6404584
       macro avg       0.96      0.89      0.92   6404584
    weighted avg       1.00      1.00      1.00   6404584
    
    
!\textbf{Confusion Matrix}!

[[  13082      17    3957       0       0       2       2]
 [     17   32454    6994       0       0      10       1]
 [    649    3821 4565771      51       2     161    1751]
 [      0       0     113     676       0       0       0]
 [      0       0       2       0 1649953       0       0]
 [      5      20     707       0       0    2833       0]
 [      4       0    2723       0       0       0  118806]]
\end{lstlisting}

\newpage
\subsubsection{RF Model ID 4 - Raw Metrics}
\begin{lstlisting}[escapechar=!]
!\textbf{S-CV Results}!
Mean AUC = 99.95
Mean F1 = 95.23
Mean Precision = 98.50
Mean Recall = 92.96
Mean Accuracy = 92.96
Training Time = 10147 seconds

!\textbf{Final Test Results}!
Weighted AUC: 0.9994792868436975
Weighted Precision: 0.984757273176817
Weighted Recall: 0.925409987596384
Weighted F1: 0.9496738038716113
Accuracy: 0.925409987596384

!\textbf{Classification Report}!

                     precision    recall  f1-score   support

          Botnet       0.14      0.96      0.24     17060
         Malware       0.22      0.97      0.36     39476
          Normal       1.00      0.90      0.95   4572206
   SQL_Injection       0.01      0.99      0.02       789
            SSDP       1.00      1.00      1.00   1649955
             SSH       0.04      0.99      0.08      3565
Website_spoofing       0.61      0.98      0.75    121533
           
        accuracy                           0.93   6404584
       macro avg       0.43      0.97      0.48   6404584
    weighted avg       0.98      0.93      0.95   6404584
    
    
!\textbf{Confusion Matrix}!

[[  16326      90      30      91       0     504      19]
 [     76   38392      13     170       0     824       1]
 [ 102163  135709 4098890   76381       2   83351   75710]
 [      0       0       1     785       0       3       0]
 [      0       0      10       0 1649945       0       0]
 [      6      15       4      16       0    3523       1]
 [    282    1145     484     262       0     355  119005]]
\end{lstlisting}


\newpage
\subsubsection{RF Model ID 5 - Raw Metrics}
\begin{lstlisting}[escapechar=!]
!\textbf{S-CV Results}!
Mean AUC = 0.9987
Mean F1 = 0.9153
Mean Precision = 0.9842
Mean Recall = 0.8665
Mean Accuracy = 0.8665
Training Time 4632.155310869217 seconds

!\textbf{Final Test Results}!
Weighted AUC: 0.998635554230961
Weighted Precision: 0.9848384599880898
Weighted Recall: 0.8600619493787575
Weighted F1: 0.9116563847511311
Accuracy: 0.8600619493787575

!\textbf{Classification Report}!

			  precision    recall  f1-score   support

      Botnet       0.06      0.94      0.12     17060
     Malware       0.10      0.90      0.19     39476
      Normal       1.00      0.81      0.89   4572206
         SQL       0.01      0.99      0.01       789
        SSDP       0.99      1.00      1.00   1649955
         SSH       0.02      0.99      0.04      3565
    WebSpoof       0.76      0.92      0.83    121533

    accuracy                           0.86   6404584
   macro avg       0.42      0.94      0.44   6404584
weighted avg       0.98      0.86      0.91   6404584
    
    
!\textbf{Confusion Matrix}!

[[  16064     141      26     133       0     693       3]
 [     72   35584     121     125       0    3572       2]
 [ 240322  301805 3690025  138640   11696  154013   35705]
 [      0       0       0     783       0       5       1]
 [      0       0       9       0 1649946       0       0]
 [      6       1       0      12       0    3546       0]
 [   4970    1711     301    1392       0     768  112391]]
\end{lstlisting}

% XGBOOOOOOOOOST

\newpage
\subsection{XGBoost}
\label{appx:XGBoost}

\subsubsection{Stock 100\% - XGBoost Raw Metrics}
\begin{lstlisting}[escapechar=!]
!\textbf{Final Test Results}!

Weighted AUC: 99.99
Weighted F1: 99.65
Weighted Precision: 99.65
Weighted Recall: 99.65
Accuracy: 99.65

!\textbf{Classification Report}!

              precision    recall  f1-score   support

           0       0.96      0.74      0.84     17060
           1       0.86      0.85      0.86     39476
           2       1.00      1.00      1.00   4572206
           3       0.93      0.88      0.90       789
           4       1.00      1.00      1.00   1649955
           5       0.95      0.79      0.86      3565
           6       0.99      0.97      0.98    121533

    accuracy                           1.00   6404584
   macro avg       0.95      0.89      0.92   6404584
weighted avg       1.00      1.00      1.00   6404584
    
    
!\textbf{Confusion Matrix}!

[[  12703      31    4320       0       0       2       4]
 [     17   33596    5857       0       0       6       0]
 [    539    5289 4564546      56       0     144    1632]
 [      0       0      91     698       0       0       0]
 [      0       0       0       0 1649955       0       0]
 [      5      15     745       0       0    2800       0]
 [      1       4    3344       0       0       0  118184]]
\end{lstlisting}

\subsubsection{Stock 100\% - XGBoost CM}
\includegraphics[width=\textwidth]{Appendices/Images/XGB/xgb_stock_100_cm.png}


\subsubsection{Stock 100\% - XGBoost Feature Importance}
\includegraphics[width=\textwidth]{Appendices/Images/XGB/xgb_stock_100_feature_importance.png}
\newpage



\section{Neural Networks}
\label{appx: Neural Networks}

\subsection{MLP NN v1}
\label{appx: MLP NN v1}

\begin{lstlisting}[language=Python]
# Create a sequential model
model = Sequential()
input_shape = (X_train_ohe.shape[1],)

# Add layers to the model
model.add(Dense(128, activation='relu', input_shape=input_shape))
model.add(Dense(64, activation='relu'))
model.add(Dense(7, activation='softmax'))

# Compile the model
model.compile(loss='categorical_crossentropy', optimizer='adam', metrics=['accuracy'])

# Train the model
model.fit(X_train_ohe, y_train_ohe, epochs=10, batch_size=32, validation_data=(X_test_ohe, y_test_ohe))

# Evaluate the model using test data
test_loss, test_acc = model.evaluate(X_test_ohe, y_test_ohe)

print('Test accuracy:', test_acc)
\end{lstlisting}

\subsubsection{MLP Neural Network}
\label{appx: MLP NN}

\begin{lstlisting}[language=Python]
from keras.models import Sequential
from keras.layers import Dense, Dropout, BatchNormalization
from keras.optimizers import SGD
from keras.initializers import he_uniform
from keras.metrics import AUC

# Define the number of classes
num_classes = 7

# Define the model architecture
model = Sequential()

# Add the input layer
model.add(Dense(100, input_shape=(X_train_ohe.shape[1],), activation='relu', kernel_initializer=he_uniform()))

# Add batch normalization
model.add(BatchNormalization())

# Add the first hidden layer
model.add(Dense(80, activation='relu', kernel_initializer=he_uniform()))
model.add(Dropout(0.25))
model.add(BatchNormalization())

# Add the second hidden layer
model.add(Dense(60, activation='relu', kernel_initializer=he_uniform()))
model.add(Dropout(0.2))
model.add(BatchNormalization())

# Add the third hidden layer
model.add(Dense(40, activation='relu', kernel_initializer=he_uniform()))
model.add(BatchNormalization())

# Add the fourth hidden layer
model.add(Dense(20, activation='relu', kernel_initializer=he_uniform()))
model.add(BatchNormalization())

# Add the output layer
model.add(Dense(num_classes, activation='softmax'))

# Define the optimizer
sgd = SGD(lr=0.01, momentum=0.9)

# Compile the model
model.compile(loss='categorical_crossentropy', optimizer=sgd, metrics=[AUC()])

# Train the model
batch_size = 170
epochs = 10
history = model.fit(X_train_ohe, y_train_ohe, batch_size=batch_size, epochs=epochs, validation_data=(X_test_ohe, y_test_ohe))

# Evaluate the model on your test data
test_loss, test_auc = model.evaluate(X_test_ohe, y_test_ohe)
\end{lstlisting}


\newpage

\end{appendices}



\subsection{Evaluation Metrics}

A key area of the work was deciding the specific metrics use to evaluate the performance of the models. Metrics are vital to determine if models were under or over-fitting on our data and helps to provide context into steps and modifications needed to improve the performances of our models. 
As a multi-class classification problem, we concerned on primarily two main metrics of evaluation: 


\subsubsection*{AUC-ROC}

The Area Under the Receiver Operating Characteristic Curve (AUC-ROC) measures the ability for a model to correctly distinguish between positive and negative classes. AUC-ROC is also insensitive to class imbalances. Similarly in the works carried in \parencite{s22155633};\parencite{pick_quality_over} AUC was used as one of the primary evaluation metrics.

\medskip

This value is first calculated by plotting the Receiver Operating Characteristic (ROC) curve using the True Positive Rate (TPR) against the False Positive Rate (FPR) for each classification thresholds. The TPR is measure of the proportions of positive values that were correctly classified. Similarly, the FPR is the proportion of negative values that are incorrectly classified as positive. Using the ROC curve, the area under the curve (AUC) is calculated. This value ranges between 0 and 1, where 0.5 represents at best random guessing, and 1 corresponds a perfect classifier.

\medskip

As our problem is multi-class, the AUC will be calculated by computing the one-vs-all metric for each class separately i.e,  calculated for each class individually, treating all samples for that class as positive and all other as negative. Then these scores are averaged to calculate a final AUC score.

\subsubsection*{F1}

The F1 score is a weighted average of both precision and recall. Precision is the fraction of correctly predicted positive instances out of all total predicted positive instances. Recall is the fraction of correctly predicted positive instances out of the total actual positive instances.

The F1-score was chosen due to its representation in an imbalanced dataset; as it considers both precision and recall. Accuracy can be a misleading metric %(\citealp[]{FAWCETT2006861}; \citealp[]{grandini2020metrics}). A model can predict the majority class i.e 'Normal' in most cases and still receive high accuracy, but in reality it poorly represents the minority classes.

\subsubsection*{Equations for Precision, Recall \& F1} 

\begin{equation*} Precision = \frac{True\ Positive}{True\ Positive + False\ Positive} \end{equation*}

\begin{equation*} Recall = \frac{True\ Positive}{True\ Positive + False\ Negative} \end{equation*}

\begin{equation*}
F_1 = 2 \cdot \frac{\mathrm{Precision} \cdot \mathrm{Recall}}{\mathrm{Precision} + \mathrm{Recall}}
\end{equation*}

\subsubsection*{Micro, Macro and Weighted}

In regular binary classification, metrics such as F1, Precision, Recall and AUC can be calculated easily, however for our multi-class classification problem a slightly different approach must be taken. In particular, there are three main methods:

\begin{itemize}
    \item Micro averaging uses the metric across all classes by counting the total true positives, false positives, and false negatives. This is the equivalent of using the accuracy i.e, fails to take into account class imbalances.
    \item Macro averaging calculates the metric in each class independently and then averages this for all classes, giving equal weight for all classes. It is used typically when all classes are equally as important, irrespective of the class size or any imbalances.
    \item Weighted averaging also calculates the metric for each class independently, but the average of the individual class scores are weighted with the number of samples in each class. It is used when performance across all classes are considered important, and the class imbalance needs to be considered.
\end{itemize}

Therefore, the weighted averaging method was chosen, leading to robust scores that takes into account both the number of samples within the class and its performance. It was observed that most previous works fails to mention the averaging method used for its evaluation metrics.

\subsubsection*{Classification Report}

In addition to viewing the averaged metrics across all classes, the classification report provides a comprehensive summary of detailing the metrics for Precision, Recall, Accuracy and F1 across each class. 

\subsubsection*{Confusion Matrix}

The Confusion Matrix is a table that displays the performance of a model by showing the number of true positives, false positives, true negatives and false negatives for each class. In other words, how accurate the classifier is on each class and how it tends to wrong predict each class for another (confusion). By examining the confusion matrix, we can identify any specific classes that may require additional tuning or changes to the model to improve its performance. Works by \citeauthor{pmlr-v29-Koco13} introduced a new method using confusion matrices to measure and analyse the performance of cost-sensitive methods showing the importance of the confusion matrix in imbalanced data sets.


\end{document}
