%!TEX root =  ../Report.tex
% %Here are some subsections so that they will appear on the contents
\subsection{Classifiers}

%Classification on the combined set of features was conducted against three classifier models, Random Forest (RF), XGBoost and K-Nearest Neighbour (KNN).

\subsubsection{Random Forest (RF)}
\label{sec:RF}

Initial modelling began using stock parameters without changing or adding any values, due to the nature of training time, we did not utilise GridSearchCV for hyperparameter tuning. Instead, we used a series of settings and parameters to get an optimal model. Each model was trained on the entire training dataset to gauge initial performance, before using 10-fold Stratified Cross Validation and evaluating finally on the test set. Table \ref{tab:rf-params} shows the parameters we considered during the creation and optimisation of our models. For the context of the device, the majority of Random Forest models were trained using the VM.

\begin{table}[h]
\centering
\caption{Parameters for Random Forest Classifier}
\label{tab:rf-params}
\begin{tabular}{|l|l|l|}
\hline
\textbf{Parameter} & \textbf{Description} \\ \hline
n\_estimators & The number of trees in the forest\\
criterion & Function to measure the quality of a split.\\
max\_depth & Maximum depth of the tree. \\
min\_samples\_split & Minimum samples required to split an internal node. \\ 
min\_samples\_leaf & Minimum samples required to be at a leaf node. \\
max\_features & Maximum features to consider when splitting. \\
bootstrap & To bootstrap samples when constructing trees \\
class\_weight & Weights associated with classes \\
random\_state & The random seed.\\ \hline
\end{tabular}
\end{table}

% \begin{table}[h]
% \centering
% \caption{RF Model Metrics}
% \label{tab:rf-metrics}
% \begin{tabular}{|l|l|l|l|l|l|l|}
% \hline
% \textbf{Model} & \textbf{Data Subset} & \textbf{Accuracy} & \textbf{Precision} & \textbf{Recall} & \textbf{F1} & \textbf{Time} \\ \hline
% Base & 80\% & 0.997 & 0.955 & 0.882 & 0.916 & 00:02:34:59 \\ \hline
% Base & 100\% & 0.997 & 0.997 & 0.997 & 0.997 & 00:00:30:60 \\ \hline
% \end{tabular}
% \end{table}

%\begin{table}[h]
%\centering
%\caption{RF Evaluation Set Metrics}
%\label{tab:rf-metrics}
%\begin{tabular}{|l|l|l|l|l|l|}
%\hline
%\textbf{Model} & \textbf{Data Size} & \textbf{Accuracy} & \textbf{Precision} & \textbf{Recall} & \textbf{F1}  \\ \hline
%Stock & 80\% & 0.997 & 0.955 & 0.882 & 0.916  \\ \hline
%Stock & 100\% & 0.997 & 0.997 & 0.997 & 0.997 \\ \hline
%\end{tabular}
%\end{table}

\begin{table}[h]
\centering
\caption{RF S-CV Mean Metrics}
\label{tab:rf-scv-metrics}
\begin{tabular}{|l|l|l|l|l|l|l|}
\hline
\textbf{Model} & \textbf{Size} & \textbf{AUC} & \textbf{F1} & \textbf{Precision} & \textbf{Recall} & \textbf{Accuracy}  \\ \hline
Stock & 100\% & 99.99 & 99.66 & 99.66 & 99.67 & 99.67 \\ \hline
Model 1 & 100\% & 99.99 & 99.66 & 99.66 & 99.67 & 99.67 \\ \hline
Model 2 & 100\% & 99.95 & 95.23 & 98.50 & 92.96 & 92.96 \\ \hline
Model 3 & 100\% & 99.87 & 91.53 & 98.42 & 86.65 & 86.65 \\ \hline
\end{tabular}
\end{table}


\begin{table}[h]
\centering
\caption{RF Evaluation Set Metrics}
\label{tab:rf-eval-metrics}
\begin{tabular}{|l|l|l|l|l|l|l|}
\hline
\textbf{Model} & \textbf{Size} & \textbf{AUC} & \textbf{F1} & \textbf{Precision} & \textbf{Recall} & \textbf{Accuracy}  \\ \hline
Stock & 100\% & 99.99 & 99.66 & 99.66 & 99.67 & 99.67 \\ \hline
Model 1 & 100\% & 99.99 & 99.66 & 99.66 & 99.67 & 99.67 \\ \hline
Model 2 & 100\% & 99.95 & 95.23 & 98.50 & 92.96 & 92.96 \\ \hline
\end{tabular}
\end{table}


\begin{table}[h]
\centering
\caption{RF Model Parameters}
\label{tab:rf-parameters}
\begin{tabular}{llllll}
\hline
Parameter & Model 1 & Model 2 \\ \hline
n\_estimators: & 100 & 200 \\
max\_depth: & 10 & 15  \\
min\_samples\_leaf: & 2 & 1  \\
min\_samples\_split: & 3 & 2  \\
random\_state: & 1234 & 1234 \\
class\_weight: & None & balanced \\ \hline
\end{tabular}
\end{table}

%\paragraph{Confusion Matrix}
%
%\begin{figure}[H]
%    \centering
%    \includegraphics[width=0.85\textwidth]{Appendices/NN Confusion Matrix 3-04-23.png}
%    \caption{RF Confusion Matrix}
%    \label{fig:rf_confusion_matrix}
%\end{figure}
