% !TEX root =  ../Report.tex

\section{Introduction}
\label{sec:Introduction} 

The ongoing increase in IoT devices in homes and commercial environments has seen a
surge in wireless networks, particularly IEEE 802.11 networks, commonly referred to as Wi-Fi. As businesses and consumers seek to try out these new devices and technologies, manufacturers tend to prioritise improving performance and features, neglecting security \parencite{roundy_iot_nodate}. As a result, this may weaken the security posture of an organisation or home to be more susceptible to attacks from malicious threat actors taking advantage of vulnerable devices within a network.

\subsection{Wireless Networks And Attacks}
\smallskip

The 802.11 standards have advanced and improved since their inception in 1997 in terms of security; however, despite this, Wi-Fi networks are still vulnerable to well-known attacks such as de-authentication attacks (to disconnect all devices from a network), leading to more advanced attacks such as Man-in-the-middle attacks (MITM) or Evil Twin attacks \parencites{DBLP:journals/corr/abs-2105-15103}. The introduction of Protected Management Frames (PMF) in 2009 \parencite{5278657} helped to increase the security of management frames by using cryptography and integrity protection on de-authentication, disassociation and action management frames \parencite{9249426}. 

The introduction of WPA3 in 2018 \parencite{wifialliance_2022_wpa3} aimed to succeed WPA2, bringing new features and fixes to strengthen the security of wireless networks. More notably, Simultaneous Authentication of Equals (SAE) was introduced to provide a secure key negotiation and key exchange method based on the Dragonfly key exchange protocol in RFC 7664 \parencite{rfc7664}, preventing dictionary or brute-forcing attacks as well as the (KRACK) Key-Reinstallation attack \parencite{krack} by providing perfect forward secrecy, ensuring that even if the private key is obtained, the data packets cannot be decrypted. 

Research into WPA3 networks indicates that even features such as PMFs and SAE authentication methods have shortcomings, including being vulnerable to denial-of-service, side-channel, and downgrade attacks \parencite{vanhoef-sp2020-dragonblood}.


% \subsection{Background}
% \label{sec:Background}

\subsection{Intrusion Detection Systems}

\smallskip

Intrusion Detection Systems (IDS) are a common mechanism to defend against these attacks by analysing network traffic and determining if they are malicious or benign. There are typically two types of intrusion detection: signature-based and anomaly-based. Signature-based IDS monitors the network traffic for any suspicious patterns within data packets that match a known signature for an intrusion. This is usually via a database holding known intrusion attack patterns. Anomaly-based IDS creates an organisational benchmark of 'normal' as a baseline to help determine whether an activity is considered unusual or suspicious. This involves initially feeding the system with a large amount of data to learn an environment's regular usage patterns. 

External tools such as Stratosphere IPS (SLIPS) developed by \textcite{garcia_2015_slips} at the Stratosphere Lab at CTU University of Prague seek to utilise a combination of behaviour patterns and machine learning such as Markov Chain models to detect malicious network traffic. Open-source implementations of wireless IDS such as Kismet \parencite{kismet_2002_kismet} and OpenWIPS-ng \parencite{thomasdotreppe_2011_openwipsng} also exist and serve a usage for both consumers and businesses.

\medskip
Significant work and research have been seen recently investigating and developing wireless intrusion detection systems using machine learning-based algorithms utilising supervised, unsupervised and deep learning approaches in wired and wireless networks. However, research on Intrusion Detection Systems utilising 802.11 and other network protocol features, e.g. ARP, TCP \& UDP, including application layer features such as HTTP, DNS, SMB etc., lacks sufficient research.

This research seeks to investigate and evaluate different machine learning algorithms in detecting and classifying attacks launched at the application layer level on 802.11 wireless networks for a proposed wireless intrusion detection system. 

\subsection{Research Questions and Objectives}
\label{sec:Research Question}

The objectives for the project are as follows:
\begin{itemize}
\item To explore and analyse current literature and academic research utilising machine learning for intrusion detection systems for IEEE 802.11 networks.
\end{itemize}
\begin{itemize}
\item To examine and identify common machine learning algorithms used for the classification in the context of network attacks.
\end{itemize}
\begin{itemize}
\item To train a combination of supervised machine learning models to classify and detect a series of attacks launched from the application layer on 802.11 wireless networks.
\item To compare the performance of such models on the dataset, proving a recommendation for a proposed Wireless Intrusion Detection System (WIDS)
%\item How does combining 802.11 specific and non-802.11 with application layer features affect the detection of application layer attacks?
%\item To utilise a published data set to create machine learning models to help classify network attacks.
\end{itemize}

%\begin{itemize}
%\item To compare and evaluate the performance of the machine learning models to provide a recommendation based on accuracy, efficiency and suitability to help develop an Intrusion Detection System (IDS).
%\end{itemize}


\section{Literature Review}                               
\label{sec: Literature Review}

This section covers the existing research and reviews literature, papers and reports focusing on publicly available datasets, existing work and different machine learning algorithms. The literature reviewed details some of the methodologies and techniques used to develop existing models for detecting network attacks on 802.11 wireless networks. The following papers and literature inspire the practical element of this project.

\subsection{Datasets}

\textcite{9664737} discusses 37 public datasets, their suitability for building and training an IDS, and their limitations and restrictions. It was concluded that these datasets do not represent newer threats, such as zero-day attacks. An optimal dataset should consist of well-labelled, up-to-date, public network traffic ranging from regular user activity to attacks and payloads. It was proposed that using multiple datasets in different network environments and scenarios across a standard set of features could help to improve the accuracy of ML-based Network Intrusion Detection Systems.

\smallskip
The UNSW-NB15 dataset, \parencite{7348942} created by The University of New South Wales in Australia, is a well-known network intrusion detection dataset consisting of 49 features with nine attack classes, specifically: Analysis, Fuzzers, Worms, DoS, Reconnaissance, Generic, Exploits, Shellcode and Backdoors. It seeks to replace older datasets such as KDD98, KDDCUP99, and NSLKDD, frequently used to evaluate NIDS. However, the dataset was generated on non-wireless hardware and therefore did not align with the requirements of a wireless network dataset.
