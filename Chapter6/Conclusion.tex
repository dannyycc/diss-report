% !TEX root =  ../Report.tex

\section{Conclusion}
\label{sec: Conclusion}

% \lipsum[7]

\subsection{Summary Of Findings}

The research proposed the idea of utilising a set of application layer features together with 802.11 and non-802.11 features to develop machine learning models capable of distinguishing between six different attacks (Botnet, Malware, SQL Injection, SDDP, SSH and Website Spoofing) launched from the application layer from the AWID3 dataset. In total, 15 application layer features, detailed in Table \ref{tab:application_features}, were chosen to be used to help form the feature set used in each model. Two classifiers: Random Forest and XGBoost and one deep neural network: Multilayer Perceptron machine learning algorithms were used to evaluate their effectiveness on the problem. 

\medskip
Random Forest showed exceptional performance with default parameters, the AUC and F1 scores on the test set were incredibly high at 99.99\% and 99.65\%, providing a great balance of precision and recall. It was able to almost completely distinguish the SDDP class from the others. Naturally, due to the imbalance of the dataset, it struggled with misclassifications of minority classes and in particular, Botnet, Malware, SQL and SSH with recall values ranging from 0.77-0.82. 

\smallskip
XGBoost, an ensemble method based on gradient boosting carried similar results to RandomForest. Metrics of the highest achieving model on the test set shared similar performances. In particular, it achieved perfect classification on the SDDP class, with zero misclassifications and the highest recall on the SQL Injection class. Like RF, it faced difficulty in detecting Botnet, Malware and SSH.

\smallskip
The MLP models showed a good overall performance but were significantly behind the other two models. The highest-performing model consisted of 3 hidden layers with dropout, batch normalisation and ReLU as the activator. The AUC, F1, Recall and Precision were high, but comparatively lower than XGBoost and RandomForest models. Like the two other models, it also struggled with under-represented classes. However, the MLP model in particular struggled to identify the SQL Injection class, with a misclassification rate of 63\%, a recall of 0.37. 

Overall, this dissertation shows that implementing application layer features with 802.11 and non-802.11 features achieved impressive performance with AUC, F1, Precision, Recall and Accuracy all above 99\% in all chosen ML algorithms. 

%Summarize the main findings of your research, including the performance of your machine learning models, insights gained from the analysis, and any limitations or areas for future work.


\subsection{Objectives}


This dissertation set out with the aim to meet a series of objectives: 

\begin{itemize}
\item To explore and analyse current literature and academic research utilising ML for intrusion detection systems for IEEE 802.11 networks.
\end{itemize}
\begin{itemize}
\item To examine and identify common machine learning algorithms used for the classification in the context of network attacks.
\end{itemize}
\begin{itemize}
\item To train a combination of supervised machine learning models to classify and detect a series of attacks launched from the application layer on 802.11 wireless networks.
\item To compare the performance of such models on the dataset, proving a recommendation for a proposed Wireless Intrusion Detection System (WIDS)
\end{itemize}


\subsubsection*{Current Literature}

\subsubsection*{Common Machine Learning Algorithms}

\subsubsection*{Supervised Machine Learning Models}
Using supervised machine learning techniques, specifically Random Forest, XGBoost and Multi-layer Perceptron, models were trained and tested. For each machine learning algorithm, a systematic approach was followed by establishing a baseline model with default parameters and then parameters and settings were tuned and changed in the hopes of optimising the performance. 

Each model subsequently achieved high levels of performance, with metrics such as AUC and F1 achieving scores of up to 99.9\% indicating a strong ability to classify the six attack classes, launched from the application layer on an 802.11 wireless network dataset, achieving the objective.

\subsubsection*{Model Performance}

\subsection{Contributions}

% Discuss the contributions of your research to the field of machine learning. This could include discussing how your research fills a gap in the literature, how it provides new insights into the problem you addressed, or how it advances the methodology used in the field.

This dissertation focused on classifying six application layer attacks on an 802.11w wireless network dataset, specifically AWID3. The research builds upon similar work but combines a selection of application layer features chosen from the 254 features total alongside 802.11 and non-802.11 features from \parencite{s22155633}. 3 machine learning algorithms: Random Forest, XGBoost and Multi-Layer Perceptron were compared and evaluated on a series of metrics, providing a valuable understanding of the strengths, weaknesses and suitability of their use in a Wireless Network Intrusion System. 

\subsection{Limitations}

%Discuss the limitations of your research and potential sources of error. Explain how these limitations could impact the generalizability of your findings and suggest ways to address these limitations in future research.

As with all research, this dissertation experience limitations and struggles that must be considered during interpretation. Due to the scope of the research, the hardware and time constraints impacted the complexity and range of exploration in the number of models, algorithms and parameters to be developed and tested. As such, hyperparameter tuning was limited and the use the grid searching techniques such as GridSearchCV and RandomisedSearchCV was limited, resorting to a more iterative and exploratory method, potentially limiting the optimal parameters for each model.

A big limitation of this research was the reliance on a single dataset, whilst the AWID3 dataset is well-established and used, the research and models conducted may not be reflective of real-world traffic and datasets with different distributions of data.

In classification problems, imbalanced datasets can lead to biased models, for the purposes of this research, no data balancing techniques such as Under/Oversampling were used. Additionally, the focus was placed on achieving a balance between specificity and sensitivity, both factors that are important in an Intrusion Detection System, therefore other interpretations may draw different conclusions in model selection.

Finally, this research did not factor in metrics such as processing time, as this can be a crucial factor to consider when being deployed in the real world.

\subsection{Future Work}

Additional work can explore the realms of including additional or different application layer features to determine their impacts on models. Additionally, future work may address the limitations encountered during this research, using more advanced neural networks such as Convolutional Neural Networks (CNNs) or Stacked AutoEncoders and utilising larger parameter grids for hyper-parameter tuning. Moreover, using data balancing techniques and feature engineering could potentially be used to provide valuable information for wireless network intrusion detection. 

Using the models in the real world would serve as a major milestone for future work, existing literature and research primarily focuses on the performance of curated datasets rather than in a network environment. However, the contributions of this research has allowed for plethora of future work to be conducted.