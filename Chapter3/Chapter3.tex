%!TEX root =  ../Report.tex

\section{Methodology}                               
\label{sec: Methodology}

% \lipsum[3]

\subsection{Code Environment}

The code for developing the machine learning models were programmed using Python 3.8/9 and Visual Studio Code and Jupyter Notebooks for the IDE. All experiments were conducted on a hardware combination of a M2 Mac Mini with 8 Cores and 16GB RAM or an Intel(R) Xeon(R) CPU E5-2699 VM running Ubuntu 22.04.02 lTS with 32 GB RAM and an Nvidia Tesla M40. Due to the limitations and errors encountered we did not utilise TensorFlow GPU Acceleration for Deep Learning on the M2 Mac Mini.

\medskip
In order to create a reproducible environment and manage dependencies, Conda virtual environments \parencite{anaconda} were used to isolate the experiments on the M2 Mac Mini. A TensorFlow GPU docker container running Nvidia CUDA was utilised on the Intel Xeon machine. See Appx \ref{appx: Conda_Env} for the full code for creating the environments.

\subsection{Libraries}

Several libraries were used to develop and implement the machine learning models, including: 
A selection of common machine learning libraries were utilised for this project, namely Numpy, Pandas, Scikit-Learn \parencite{scikit-learn}, Matplotlib, Seaborn, Joblib, Jupyter, Tensorflow \parencite{tensorflow2015-whitepaper} and XGboost \parencite{XGBoost}. 




