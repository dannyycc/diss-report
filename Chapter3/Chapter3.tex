%!TEX root =  ../Report.tex

\section{Methodology}                               
\label{sec: Methodology}

% \lipsum[3]


\subsection{AWID3 Dataset}


\begin{itemize}
	\item SSH Bruteforce - a brute-force attack was conducted against the radius server 
	\item Botnet - The attack deployed a piece of malware and infects a number of STAs, once executed the victim's machine pings back to the C2 server. 
	\item Malware - Malware was launched that aims to exploit a specific vulnerability in the target system to further escalate the attack, e.g., deploy ransomware.
	\item SQL Injection - The target is an external node and a malicious SQL query string is inputted into a web form of the target. The packet's HTTP POST and GET requests can reveal the SQL code query.
	\item SDDP Amplification - This attack consists of a DDoS attack using the Simple Service Discovery Protocol and uses Universal Plug and Play (UPnP) to trick all STAs of the wireless network to send a barrage of packets to each SDDP-enabled device. Every device then responds, eventually leading to a DoS.
\end{itemize}

\subsection{Code Environment}

The code for developing the machine learning models was programmed using Python 3.8/9, Visual Studio Code, and Jupyter Notebooks for the IDE. All experiments were conducted on a hardware combination of an M2 Mac Mini with 8 Cores and 16GB RAM or an Intel(R) Xeon(R) CPU E5-2699 VM running Ubuntu 22.04.02 LTS with 64 GB RAM and an Nvidia Tesla M40. Accordingly, the two machines will be referred to as 'M2' and 'VM'. Due to the limitations and errors encountered, TensorFlow GPU Acceleration was not utilised for Deep Learning on the M2 Mac Mini.

\medskip
In order to create a reproducible environment and manage dependencies, Conda virtual environments \parencite{anaconda} were used to isolate the experiments on the M2 Mac Mini. A TensorFlow GPU docker container running Nvidia CUDA was utilised on the VM. See Appx \ref{appx: Conda_Env} for the full code for creating the environments.

\subsection{Libraries}

Several libraries were used to develop and implement the machine learning models, including: 
A selection of common machine learning libraries was utilised for this project, namely Numpy, Pandas, Scikit-Learn \parencite{scikit-learn}, Matplotlib, Seaborn, Joblib, Jupyter, Tensorflow \parencite{tensorflow2015-whitepaper} and XGBoost \parencite{XGBoost}. 




